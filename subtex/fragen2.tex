\begin{center}
  \begin{tabular}{ll|ll|ll}
   \toprule
   Frage & Sicherheitsgrad & Frage & Sicherheitsgrad 
         & Frage & Sicherheitsgrad \\
   \midrule
   1& Sehr sicher      & 6 & Sicher           & 11& Sehr sicher\\ 
   2& Sehr sicher      & 7 & Sehr sicher      & 12& Sehr sicher\\
   3& Sehr sicher      & 8 & Sehr sicher      & 13& Sehr sicher\\
   4& Sehr sicher      & 9 & Sicher           & 14& Sicher\\
   5& Sicher           & 10& Sicher           & 15& Sicher\\
   \bottomrule
  \end{tabular}
\end{center}
\begin{enumerate}
  \item Wie lautet die Taylor-Entwicklung von $\rr{1}$ in den Komponenten
        von $\vb* r'$ bei $\vb* r'=0$ bis einschließlich der Zweiten
        Ordnung?
        $$ 
        \rr{1}\approx \frac{1}{r}+\frac{\vb* r\cdot\vb* r'}{r^3} +
      \frac{1}{2}\sum_{i,j=1}^3\frac{3r_{i}r_{j}-\delta_{ij}\vb*r^2}{r^5}
        r_i'r_j'
        \quad\text{(Skript 1.6.2)}$$
        \textbf{Notiz 1:} Dies ist eine ``Fernfeld'' Näherung, d.h. 
        eine Gute Näherung für $r\gg 0$ (denn für $r=0$ divergiert
        die Näherung).\\
        \textbf{Notiz 2:} Für genauere Information zur mehrdimensionale
        Taylor-Entwicklung, schaue im HöMa II Skript oder schaue hier 
        nach:~\url{https://de.wikipedia.org/wiki/Taylorreihe#Mehrdimensionale_Taylorreihe}

  \item Man drücke das elektrostatische Potential $\Phi(\vb* r)$ einer
        räumlich begrenzte Ladungsverteilung $\rho(\vb* r')$ mittels
        Monopol, den Dipolvektor und den Quadrupoltensor aus.
        $$
        \Phi(\vb* r)\approx k\frac{Q_\text{ges}}{r} +
        k\frac{\vb* r\cdot \vb* p}{r^3} +
        \frac{k}{2} \sum_{ij}\frac{r_ir_j}{r^5}Q_{ij}
        \smref{1.6.6}$$

  \item Man gebe Formel für das Monopolmoment, den Dipolvektor und den
        Quadrupoltensor an.
        \begin{center}
        \begin{tabular}{lll}
          Monopol:    & $\ds Q_{\text{ges}}=\int d^3r\rho(\vb* r)$
                        &\sref{1.6.7}\\
          Dipol:      & $\ds p_i=\int d^3r\rho(\vb* r)r_i
                        \quad\vb* p=p_i\vu{e}_i $
                        &\sref{1.6.8}\\
          Quadrupol:  & $\ds Q_{ij}=\int d^3r\rho(\vb* r)
          \qty(3r_ir_j - \delta_{ij}\abs{\vb* r}^2)$
                        &\sref{1.6.9}\\
        \end{tabular}
        \end{center}
        \textbf{Notiz 1:} $Q_{ij}$ hat nur 5 Freiheitsgraden! D.h.\ man kann           mit nur 5 Rechnungen alle (9) Quadrupol Elemente berechnen! Es
        gilt zwar $Q_{ij}=Q_{ji}$ und 
        $\text{sp}(\bm Q)=\sum_i Q_{ii}=0$.\\
        \textbf{Notiz 2:} Die hier nicht gefragte Kugelkoordinaten
        Multipolentwicklung ist auch sehr wichtig! 
        (Siehe Skript §1.6.6 S.42)

  \item Wie koppeln Monopol und Dipol an ein externes elektrostatisches
        Potential bzw. Feld?
        
        Für eine Multipol im Zentriert um $\vb* r_0$:
        $$E_{\text{pot}}=\Phi(\vb* r_0)Q-\vb E(\vb* r_0)\cdot\vb* p 
        \qquad
        \underbrace{\qty(-\frac{1}{6}\sum_{ij}\pdv{x_j}E_i(\vb* r_0)Q_{ij}
        +\ldots)}_{\text{hier nicht gefragt!}}
        \smref{1.7.7}$$
        \textbf{Notiz:} Energie minimierung für $\vb* p\parallel \vb E$
        und gleich ausgerichtet (also $\vu e_{\vb* p}=\vu e_{\vb E}$).\\

  \clearpage
  \item Wie lautet die elektrostatische Energie einer Ladungsverteilung?
        Man gebe zwei Äquivalente Ausdrücke, einerseits mit $\rho(\vb* r)$
        und andersseits mit $\vb E(\vb* r)$.
        \begin{center}
        \begin{tabular}{lll}
          mit $\rho$    & $\ds E_{\text{WW}}=
                          \frac{1}{2}\int d^3 r\rho(\vb* r)
                          \Phi_\rho(\vb* r)$
                            &(Skript 1.7.9)\\
          mit $\vb E$   & $\ds E_{\text{WW}}=\frac{1}{8k\pi}
                           \int d^3 r\vb E^2(\vb* r)$
                            &(Skript 1.7.15)\\
        \end{tabular}
        \end{center}

  \item Wie ist die elektrische Suszeptibilität $\chi$ in einem 
        linearen, isotropen Medium definiert?
        \begin{center}
        \begin{tabular}{lll}
          Definition:             
          & $\vb P = \frac{1}{4\pi k} 
            \vu \chi(\vb E)\cdot \vb E\quad$
          & (Skript 1.8.1) \\
          Falls isotrop: 
          & $\vu\chi(\vb E)=
            \mathbb{1}\chi(\vb E)$
          & (Skript 1.8.2) \\
          Mit schwache abh.\ von $\vb E$:
          & $\chi(\vb E)\approx\chi
          \quad(\chi=const.\in\mathbb{R})$
            & (Skript 1.8.3) \\
          (lineares Medium)
        \end{tabular}
        \end{center}

        Sei die Frage genau so gestellt, so würde ich in der Klausur einfach
        das untere aufschreiben:
        \begin{equation*}
          \vb*P = \frac{1}{4\pi k}\chi \vb E 
        \end{equation*}
        Dabei würde ich angeben dass die definition im Skript analog ist,
        aber in diese (meine) schreibweise $\chi$ dimensionslos ist. 
        Man kann natürlich auch die Definition aus dem Skript direkt
        hinschreiben.

        \textbf{Notiz 1:} $\vu \chi$ ist im allgemeinsten Fall 
        definiert als ein Matrix/Tensor.
        dies lässt zu, dass $\vu \chi$ in unterschiedliche Richtungen
        andere Auswirkungen hat (Polarisation kann z.B. in x-Richtung
        Stärker sein als in y-Richtung). Wir betrachten in diesen Kurs 
        aber im Allgemein
        lineare, isotrope Medien, wobei es keine Richtungsabhängigkeit
        gibt, und wobei $\chi$ auch nicht Stark abhängt von das äußere
        $E$-Feld. Dadurch kann
        man $\chi$ als eine Medienabhängige skalare Konstante 
        sehen/nähern.\\
        \textbf{Notiz 2:} Die im Skript gegebene Definition ist 
        \textbf{Falsch}! (aber richtig bis auf der Vorfaktor.)\\

  \item Wie lautet der Zusammenhang zwischen elektrischem Feld, 
        dielektrischer Verschiebung und Polarisation?
        $$\vb E= 4\pi k\vb D - 4\pi k\vb P \quad\text{(Skript 1.8.17)}$$
        \textbf{Notiz 1:} Die im Skript gegebene Definition ist 
        \textbf{Falsch}! (aber richtig bis auf die Vorfaktoren.)
        Ich würde in der Klausur einfach die Definition in SI einheiten
        schreiben, denn die ist auch im Skript richtig angegeben!\\
        \textbf{Notiz 2:} Alternativ gilt $\vb D=\epsilon \vb E
        \stackrel{\text{(SI)}}=\epsilon_0\epsilon_r \vb E$ mit
        $\epsilon=\frac{1+\chi}{4\pi k}$ (Herleitung wie im Skript) und
        $\epsilon_r=1+\chi$
        (unter die richtige Bedingungen, siehe Skript) (Skript 1.8.17)\\

  \item Welche Feldgleichung erfüllt die dielektrische Verschiebung?
        $$\div \vb D=\rho_f(\vb* r)\quad\text{(Skript 1.8.4)}$$
        $$\qty(\div \vb P=-\rho_P(\vb* r))\quad\text{(Skript 1.8.6)}$$
        \textbf{Notiz:} Die im Skript gegebene Definition ist 
        \textbf{Falsch}! (aber richtig bis auf der Vorfaktor.)

  \item Was geschieht mit dem $1/r$-Potential von Punktladungen in
        Metallen?
        \begin{center}
          Die Punktladungen werden abgeschirmt (Potential fällt 
          exponentiell ab)
          (Skript 1.8.30)
        \end{center}
        \textbf{Notiz:} Siehe auch Yukawa Potential: 
        \url{https://en.wikipedia.org/wiki/Yukawa_potential} 

  \clearpage
  \textbf{[MAGNETOSTATIK]}
  \item Man schreibe die Kontinuitätsgleichung in differentieller Form
        und mit Integralen über ein Volumen $V$ bzw.\ dessen Oberfläche 
        $\partial V$.
        \begin{center}
        \begin{tabular}{lll}
          Differentieller Form 
                            & $\ds\pdv{t}\rho(\vb* r, t)=
                               -\div\vb* j(\vb* r, t)$
                              & (Skript 2.1.5)\\
          Integral Form     & $\ds\dot Q_V
                              =\int_V d^3r\partial_t\rho(\vb* r,t)=
                              -\int_{\partial V}d\vb* A\cdot
                              \vb* j(\vb* r, t)=-I_{\partial V}$
                              & (Skript 2.1.1 \& 3)\\
        \end{tabular}
        \end{center}
        \textbf{Notiz:} Bemerke das beim Integral form den Satz v. Gauß
        angwand wurde.

  \item Man schreibe Ladungsdichte $\rho(\vb* r)$ und Stromdichte
        $\vb* j(\vb* r)$ für $i=1,\ldots,N$ Punktladungen $q_i$ mit 
        Trajektorien $\vb* r_i(t)$.
        $$\rho(\vb* r)=\sum_i q_i\delta(\vb* r-\vb* r_i(t))
        \quad \text{(Skript 1.1.2)}$$
        $$\vb* j(\vb* r)=\sum_i q_i\dot(t){\vb* r}_i\delta(\vb* r-\vb* r_i(t))
        \quad \text{(Skript 2.1.2)}$$

  \item Wie lautet das Biot-Savart'sche Gesetz für $\vb* A(\vb* r)$ und
        $\vb B(\vb* r)$?
        $$\vb B(\vb* r)=k'\int d^3r'\frac{\vb* j(\vb* r')\times
        (\vb* r-\vb* r')}{\abs{\vb* r-\vb* r'}^3}
        \quad \text{(Skript 2.2.1)}$$
        $$\vb* A(\vb* r)=k'\int d^3r'\rr{\vb* j(\vb* r')}
        \quad \text{(Skript 2.3.2)}$$
        \textbf{Notiz 1:} $k'=\frac{\mu_0}{4\pi}$ (SI)\\
        \textbf{Notiz 2:} $\curl \vb* A(\vb* r) = \vb B(\vb* r)$ und 
        $\curl \rr{j(\vb* r')}
        =\frac{\vb* j(r') \times (\vb* r - \vb* r')}{\abs{\vb* r - \vb* r'}^3}$

  \item Wie lauten die zwei Feldgleichungen der Magnetostatik?
    \begin{equation*}
      \begin{array}{ll}
        \div \vb B(\vb* r) = 0&\smref{2.3.5}\\
        \curl \vb B(\vb* r)= 4\pi k'\vb* j(\vb* r)&\smref{2.3.12}
      \end{array}
    \end{equation*}

  \item Man berechne das magnetische Dipolmoment für eine Punktladung auf
        einer Kreisbahn mit Radius $R$ sowie Drehimpuls $\vb* L$.
        $$\vb* j(\vb* r,t)=q\vb* v(t)\delta(\vb* r-\vb* r(t))\smref{2.1.2}$$
        $$\vb* m(t) 
        = \frac{1}{2}\int d^3 r (\vb* r \times \vb* j(\vb* r,t))
        =\frac{q}{2}\vb* r(t)\times\vb* v(t)=\frac{q}{2m}\vb* L(t)$$
        \textbf{Notiz:} Definition von $\vb* m$ \sref{2.4.8}

  \item Wie lauten die Zusammenhänge zwischen $\vb B$, $\vb H$ 
        und $\vb M$ in einem Paramagneten?

        Allgemein:
        $$\vb B = 4\pi k'\vb H + 4\pi k'\vb M\quad\text{(Skript 2.5.12)}$$
        $$\vb B = \mu_0 \vb H + \mu_0 \vb M \quad\text{(SI) 
        (Skript 2.5.17)}$$
        Im (Dia- und) Paramagneten:
        \begin{equation*}
            \vb M 
            = \chi_M \vb H 
            = \frac{1}{4\pi k'}\frac{\chi_M}{1+\chi_M} \vb B
            \stackrel{\text{(SI)}}= \frac{\chi_M}{\mu_0\mu_r} \vb B
        \end{equation*}
         \begin{equation*}
            \vb B = \mu \vb H \quad \text{mit}\quad
            \mu = 4\pi k'(1+\chi_M)
            \stackrel{\text{(SI)}}= {\mu_0\mu_r} 
         \end{equation*}
          \begin{center}
            \sref{2.6.3$\sim$6}
          \end{center}

        \textbf{Notiz 1:} $\ds \curl \vb H = \vb* j_f$ und 
        $\ds \curl \vb M=\vb* j_M$\\
        \textbf{Notiz 2:} Die im Skript gegebene Definition ist 
        \textbf{Falsch}! (aber richtig bis auf der Vorfaktor.)
\end{enumerate}
