\setcounter{footnote}{0}
\begin{center}
  \begin{tabular}{ll|ll|ll}
    \toprule
    Frage & Sicherheitsgrad & Frage & Sicherheitsgrad 
          & Frage & Sicherheitsgrad \\
    \midrule
    1& Sehr sicher      & 6 & Sicher     & 11& Sicher\\ 
    2& Sehr sicher      & 7 & Sicher     & 12& Sicher\\
    3& Sicher           & 8 & Sicher     & 13& Sicher\\
    4& Sehr sicher      & 9 & Sicher     & 14& Sicher\\
    5& Sehr sicher      & 10& Sehr sicher& 15& Sicher\\
    \bottomrule
  \end{tabular}
\end{center}
\begin{enumerate}

  \item Wie lauten die vollen, makroskopischen Maxwell-Gleichungen für
        die Felder $\vb E(\vb* r, t)$ und $\vb B(\vb* r, t)$
        \begin{center}
        \scalebox{0.8}{
        \begin{tabular}{lll}
        Name          & Differentiell       
                            & Integral\\
        Gauß'scher Satz     
                      & $\ds \div \vb E(\vb* r, t) 
                        = 4\pi k\rho(\vb* r, t)$
                            & $\ds \oint_{\partial V}d\vb* A\cdot
                              \vb E(\vb* r, t)=
                              4\pi k Q_{\text{ges,V}}$\\
        Keine Magn. Monopole
                      & $\div \vb B(\vb* r, t)=0$
                            & $\ds\oint_{\partial V}d\vb* A\cdot 
                              \vb B(\vb* r, t) = 0 $\\
        Faraday'sche Gesetz 
                      & $\ds \curl \vb E(\vb* r , t)=-k''\partial_t
                        \vb B(\vb* r, t)$
                      & $\ds \oint_{\partial A}d\vb* l\cdot\vb 
                      E(\vb* r, t)=-k''\dv{t}\iint_Ad\vb* A\cdot \vb B(\vb* r, t)$\\
        Ampère'sche Gesetz  
          & $\ds \curl \vb B(\vb* r, t)=4\pi k'\vb* j(\vb* r, t)
            +\frac{k'}{k}\partial_t \vb E(\vb* r, t)$
          & $\ds\oint_{\partial A}d\vb* l\cdot \vb B(\vb* r, t)=
          \iint_{A}d\vb* A\cdot\qty(4\pi k'\vb* j(\vb* r, t)+
          \frac{k'}{k}\partial_t \vb E(\vb* r, t))$\\
        \end{tabular}
        }
        (SI: $k=\frac{1}{4\pi\epsilon_0}\quad k'=\frac{\mu_0}{4\pi}\quad
                  k''=1$)
        (Gauß: $k=1\quad k'=\frac{1}{c}\quad
             k''=\frac{1}{c}$) 
        \end{center}
        \begin{center}
          \sref{3.4.1$\sim$4}
        \end{center}
        
        \textbf{Notiz:} Die Integral-Form der Maxwell-Gleichungen folgt
        direkt aus beidseitig Integrieren der Differential-Form (über 
        Volumen für die Divergenz-Gesetze, über Oberfläche für die 
        Rotations-Gesetze) unter Anwendung der Gauß'sche und 
        Stoke'sche Gesetze.

  \item Welche Gleichung legt das Verhalten der aus Punktladungen
        zusammen gesetzten Quellen $\rho(\vb* r, t)$ und 
        $\vb* j(\vb* r, t)$ in den Feldern $\vb E(\vb* r, t)$
        und $\vb B(\vb* r, t)$ fest?
        \begin{center}
          Die allgemeine Lorentzkraft: 
          $\ds \vb* F_q=q\qty(\vb E+k''\vb* v\times\vb B)\smref{3.7.1}$
        \end{center}

  \item Man zeige mittels Kontinuitätsgleichung und Ampère-Gesetz, 
        dass der Maxwell'sche Verschiebungsstrom für Ladungserhaltung
        notwendig ist.
        $$\div \vb* j(\vb* r, t) = -\pdv{t}\rho(\vb* r ,t)$$
        $$\curl\vb B(\vb* r, t) \coloneqq 
        4\pi k'\vb* j(\vb* r ,t) + \vb* C(\vb* r, t)$$
        $$\text{Allgemein gilt:}\,\div\curl\vb* F(\vb* r, t)
        =\curl \div \vb* F(\vb* r, t)=0$$
        \begin{equation*}
          \begin{split}
          \ds \div\curl\vb B(\vb* r, t)
                      &\ds= \div \qty[
                      4\pi k'\vb* j(\vb* r, t)+\vb* C(\vb* r, t)]\\
                      &\ds= 4\pi k'\div\vb* j(\vb* r,t) +
                      \div \vb* C(\vb* r,t)\\
                      &\ds= 4\pi k'\qty(-\pdv{t}\rho(\vb* r, t)) 
                      +\div \vb* C(\vb* r,t)\stackrel{!}{=}0\\
          \Leftrightarrow\ds 4\pi k'\pdv{t}\rho(\vb* r, t) 
                      &\ds =\div \vb* 
                      C(\vb* r, t)\footnotemark
                      \stackrel{!}{=}\div \pdv{t}K\vb E(\vb* r, t)\\
                      &\ds =\ds \pdv{t} 
                      K4\pi k \rho(\vb* r, t)
          \Leftrightarrow K         
                      \stackrel{!}{=}\frac{k'}{k}\\
          \Leftrightarrow \vb* C(\vb* r,t)
                      &\coloneqq \frac{k'}{k}\pdv{t}\vb E(\vb* r, t)
          \end{split}
        \end{equation*}
        \footnotetext{Erinnere daß $\div \vb E(\vb* r, t)=4\pi k\rho(\vb* r ,t)$}      

  \item Wie lauten die Wellengleichungen für $\vb E(\vb* r, t)$ und
        $\vb B(\vb* r, t)$ im Vakuum?
        $$\qty(\Delta - \frac{1}{c^2}\pdv[2]{t})\vb E(\vb* r, t)=
        \qty(\Delta - \frac{1}{c^2}\pdv[2]{t})\vb B(\vb* r, t)=0
        \smref{3.5.2 und 3.5.3}$$

  \item Wie sehen Ebene-Wellen-Lösungen im Vakuum aus? Was weiß man
        über die Richtungen der Felder bezüglich der 
        Ausbreitungsrichtung?
        $$f(\vb* r , t)=f_r(\vb* r)f_t(t)$$
        $$\vb* k \,\bot\, \vb E \quad \vb* k\,\bot\,\vb B \quad
        \vb E\,\bot\, \vb B\smref{3.5.8$\sim$10}$$

  \item Wie erhällt man die Dispersionsrelation und wie lautet sie für
        elektromagnetische Wellen im Vakuum?
        \begin{center}
          Indem man die Lösung (gegebene Funktion) in 
          der Wellengleichung einsetzt und die
          triviale Terme gegen einander wegstreicht.
        \end{center}
        $$\dv{\omega(k)}{k}=c \quad\text{(im Vakuum)}\smref{3.5.7}$$

  \item Man gebe die Formeln für die Fourier-Hin- und Rücktransformation
        für eine Funktion $f(t)$ auf der $t$-Achse auf $\omega$-Achse und
        zurück an.
        $$\hat f(\omega, \alpha)=\int \frac{dt}{\sqrt{2\pi}}
              e^{-i\omega t}f(t, \alpha)\smref{3.6.12}$$
        $$f(t, \alpha)=\int \frac{d\omega}{\sqrt{2\pi}}
                         e^{+i\omega t}\hat f(\omega, \alpha)
                        \smref{3.6.11}$$
        \begin{center}
          ($\alpha$ stellt hier nicht zur Transformation relevante 
          sonstige Parameter dar, wie z.B. $\vb* r$)
        \end{center}
        \textbf{Notiz:} Hier wird explizit um die Fouriertransformation 
        nach Zeit gefragt. Dies darf nicht verwechseld werden mit die 
        Fouriertransformation von $\vb* x$-raum nach $\vb* k$-raum 
        wobei man über
        $d^3x$ bzw. $d^3k$ integriert wird \sref{3.6.17 und 3.6.18}.
        Welche Parameter transformiert werden hängt also von der Frage 
        ab!

  \item Man berechne die Fouriertransformation von $\delta(t-t_0)$.
        $$f(t)=\delta (t-t_0)$$ 
        $$F[f(t)]=\hat f(\omega)
        =\int\frac{dt}{\sqrt{2\pi}}e^{-i\omega t}\delta(t-t_0)
        =\frac{1}{\sqrt{2\pi}}e^{-i\omega t_0}
        \smref{3.9.7}$$ 

  \item Wie berechnet sich der Poynting-Vektor für reelle Felder? Was
        ist seine physikalische Bedeutung?
        $$\vb S(\vb* r,t)=\frac{1}{\mu_0}\vb E(\vb* r,t)\times\vb B(\vb* r,t)
        =\vb E(\vb* r,t)\times \vb H(\vb* r,t)
        \smref{3.7.8}$$
        $$\text{(Allgemein)}\quad\vb S(\vb* r,t)=
        \frac{1}{\mu_0}\Re[\vb E(\vb* r,t)]\times \Re[\vb B(\vb* r,t)]$$
        
        \begin{center}
          Die physikalische Bedeutung ist die von den 
          Feldern verursachte Energiefluß.
        \end{center}

  \clearpage
  \item Wie lauten die inhomogenen Maxwell-Gleichungen in Materie,
        charakterisiert durch Dielektrizitätskonstante $\epsilon_r$ und
        Permeabilität $\mu_r$?
        \begin{center}
          \begin{tabular}{rrcl}
            Inhomogen&$\div \vb D$ & $=$ &$\rho_f$\\
            &$\curl \vb H-\pdv{t}\vb D$ & $=$ &$\vb* j_f$\\
            Homogen&$\curl \vb E+\pdv{t}\vb B$ & $=$ &$0$\\
            &$\div \vb B$ & $=$ &$0$\\
          \end{tabular}
          \sref{4.1.5$\sim$8}
          $$\vb D(\vb* r,t)=\epsilon_r\epsilon_0\vb E(\vb* r,t) \quad
          \text{und}\quad\vb H(\vb* r, t)=\frac{1}{\mu_0\mu_r}
          \vb B(\vb* r,t)\smref{4.1.9}$$
        \end{center}

  \item Durch welche drei Sachverhalte unterscheiden sich 
        elektromagnetische Wellen in eienem Wellenleiter mit einfach
        zusammenhängenden rechteckigem Querschnitt von denen im Vakuum?
        \begin{enumerate}
          \item Die Dispersionsrelation ändert sich 
                ($\omega(\vb* k)\ne c\abs{\vb* k}$ i.A.)
          \item Es gilt i.A. \textbf{nicht} dass $\vb E\,\bot\,\vb B$
          \item Es gilt i.A. \textbf{nicht} dass $\vb E\,\bot\,\vb* k$
                oder dass $\vb B\,\bot\,\vb* k$
        \end{enumerate}
        \begin{center}
          \sref{§3.11}
        \end{center}

  \item Wie sind Phasen- und Gruppengeschwindigkeit definiert und was ist
        ihre physikalische Bedeutung?
        \begin{center}
          \begin{tabular}{ll}
            $v_{\text{ph}}=\frac{\omega(k)}{k}$ 
                      &Gescwhindigkeit mit 
                       den die Spitzen der Welle sich ausbreiten 
                       \\\\
            $v_{\text{gr}}=\dv{\omega(k)}{k}$
                      &Gewschwindigkeit mit
                       den die Knoten einer ebene Welle sich ausbreiten.\\
                      &Geschwindigkeit eines gesammten Wellenpackets.
                      \sref{3.12.9}\\
          \end{tabular}
        \end{center}
        \textbf{Notiz:} 
        $v_{\text{ph}}$ kann durchaus größer als $c$ sein. Dies ist aber
        kein Problem, denn die Phasengeschwindigkeit ``trägt'' keine
        Information. Für $v_{\text{gr}}$ gilt immer $v_{\text{gr}}\le c$!

  \item Man gebe zwei Formen für das Ohm'sche Gesetz an. Wie ist die
        Beziehung zwischen Leitfähigkeit $\sigma$ und Widerstand $R$ 
        für einen homogenen Leiter der länge $l$ und mit 
        Querschnitt $A$?
        \begin{enumerate}
          \item $U=IR\smref{3.2.1}$
          \item $\vb* j= \sigma \vb E\smref{3.2.2}$ 
            $\qty(\sigma=\frac{1}{\rho}\quad R=\frac{l}{\sigma A})$
        \end{enumerate}

  \item Wie entsteht eine komplexe, frequenzabhängige 
        Dielektrizitätskonstante $\epsilon(\omega)$ in einem Metal?
        $$\curl \vb H = \vb* j + \pdv{t}\vb D$$
        $$\curl \vb H -\qty(\sigma +\epsilon_0\epsilon_r\pdv{t})\vb E=0$$
        $$\curl \vb H -\qty(\sigma +\epsilon_0\epsilon_ri\omega)\vb E=0$$
        $$\curl \vb H +\epsilon_0 i\omega
        \qty(\epsilon_r-\frac{\sigma}{i\omega\epsilon_0})\vb E=0$$
        $$\curl \vb H +i\omega\epsilon_0 \epsilon(\omega)\vb E=0$$
        $$\epsilon(\omega)=\qty(\epsilon_r-\frac{\sigma}
        {i\omega\epsilon_0})\smref{4.2.5}$$

  \item Wie verhält sich die Dielektrizitätskonstante $\epsilon(\omega)$
        bei der Plasmafrequenz $\omega_P$? Was passiert physikalisch bei
        $\omega=\omega_P$ und bei $\omega>\omega_P$?
        $$
        \epsilon(\omega_P)=\epsilon_r\qty(1-\frac{\omega_P^2}{\omega^2})
        \smref{4.2.23}$$
        $$\Leftrightarrow\quad\epsilon(\omega_P)=0
        \quad\epsilon(\omega<\omega_P)<0 
        \quad\epsilon(\omega>\omega_P)>0$$
        $$\quad\omega_P=\sqrt{\frac{n_fe^2}
        {m\epsilon_0\epsilon_r}}\smref{4.2.31}$$
        \begin{tabular}{rl}
          $\ds\omega>\omega_p$ &$\Rightarrow$ 
          Metalle werden durchsichtig\\
          $\ds\omega\le\omega_p$& $\Rightarrow$ Exponentielle 
          Abfall von EM-Wellen im Medium
        \end{tabular}
        \begin{center}
          \sref{§4.2.3 S.123}
        \end{center}

\end{enumerate}
