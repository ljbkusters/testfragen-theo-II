\begin{center}
  \begin{tabular}{ll|ll|ll}
    \toprule
    Frage & Sicherheitsgrad & Frage & Sicherheitsgrad 
          & Frage & Sicherheitsgrad \\
    \midrule
    1& Sicher           & 6 & Sicher     & 11& Sicher  \\ 
    2& Sicher           & 7 & Sicher     & 12& Sicher  \\
    3& Sicher           & 8 & Sicher     & 13& Sicher  \\
    4& Sicher           & 9 & Sicher     & 14& Unsicher\\
    5& Sicher           & 10& Sicher     & 15&         \\
    \bottomrule
  \end{tabular}
\end{center}

  \textbf{Wichtibe bemerkung:} Dieses Kapitel arbeitet schließlich
  mit Gauß'sche Einheiten!

\begin{enumerate}
  \item Was bedeutet das spezielle Relativitätsprinzip für physikalische %1
    Gesetze und die Lichtgeschwindigkeit?

    Die Lichtgeschwindigkeit ist eine Konstante in jedem Bezugssystem.
    Die physikalische Gesetze sollen in jedes Bezugssytem gleich sein,
    unterschiedliche Beobachter können aber unterschiedliche Gründen
    angeben weshalb ein Erreignis statt findet. (z.B. kann ein Beobachter
    im Ruhesystem einer Ladung nur Elektrische Kräfte sehen, während
    ein Beobachter in ein Ruhesystem wo die Ladung sich bewegt auch
    magnetische Kräfte sieht.) \sref{§5.2.1}

  \item Wie lautet die Formel für den invarianten Raum-Zeit-Abstand ds in %2
    einem Koordinatensystem? Warum folgt aus der Invarianz von $ds$ die
    Invarianz der Lichgeschwindigkeit?
    
    Die invariante Raum-Zeit-Abstand ist gegeben durch:
    \begin{equation*}
      ds^2 = c^2 dt^2 - d\vb* x^2
      \smref{5.2.5}
    \end{equation*}

      Für ein Teilchen, wie z.B. ein Photon, dass sich mit der 
      Lichtgeschwindigkeit bewegt soll gelten dass die Minkowski-Metrik
      verschwindet. Das heißt dass:
      \begin{equation*}
        (cdt)^2 - d\vb* x^2 = 0 
      \end{equation*}
      Wobei natürlich gilt dass $d\vb* x=\vb* vdt$ mit $\abs{\vb* v}=c$.
      ($d\vb* x^2 = dx^2 + dy^2 + dz^2$)

      In System $S'$ gibt es jetzt die Parameter $t'$, $\vb* x'$ und $\vb* v'$
      mit $d\vb* x'=\vb* v'dt'$. Wir möchen zeigen, dass 
      $\abs{\vb* v}=\abs{\vb* v'}$ falls $\abs{\vb* v}=c$.
      Aus der Invarianz von $ds$ folgt dass $ds=ds'$, woraus folgt dass
      \begin{equation*}
        c^2dt^2-d\vb* x^2 = ds^2 = 0 = ds'^2 = c^2dt'^2 - d\vb* x'^2
      \end{equation*}
      Es muss also gelten, dass $c^2dt'^2 = d\vb* x'^2$ und deswegen folgt
      $c=\abs{\vb* v'}\Rightarrow \abs{\vb* v}=\abs{\vb* v'}=c$, 
      sodass sich die Geschwindigkeit 
      des Teilchens (bei Lichtgeschwindigkeit) als invariant ergibt.

  \item Welche Bedingungen stellt man an eigentliche orthochrone %3
    Lorentz-transformationen?

    \textbf{Definition:} Eine Lorentz-Transformation, bei der (Raum-) 
    Spiegelungen 
    ausgeschlossen sind und die Orientierung der Zeit erhalten ist, 
    wird als eigentliche, orthochrone Lorentz-Transformation bezeichnet.

    \begin{equation*}
      \Lambda^{0}_{\ 0} \ge 1 
      \quad \text{und}\quad
      \det \Lambda = 1 \smref{5.2.18}
    \end{equation*}
    \textbf{Notiz:} siehe auch 
    \url{https://de.wikipedia.org/wiki/Lorentz-Transformation} (suche 
    nach orthochron mit deinem Browser)

  \item Wie lautet die Lorentz-Transformation für einen Boost in %4
    $z$-Richtung? Wie sieht die Inverse aus?
    
    Sei $S$ ein Ruhesystem und bewege $S'$ sich mit relativen 
    Geschwindigkeit $v$ in $z$-Richtung. Man definiere die (einheitslosen)
    Größen:
    $\beta = \frac{v}{c}$, $\gamma=\qty(1-\frac{v^2}{c^2})^{-\frac{1}{2}}$
    \sref{5.2.45}
    
    Für die Hintransformation gilt:
    \begin{equation*}
      x'^\mu = \Lambda^\mu_{\ \nu} x^\nu 
      \text{ mit } 
      \Lambda^\mu_{\ \nu}=
      \begin{pmatrix}
        \gamma       & 0 & 0 & -\beta\gamma\\
        0            & 1 & 0 & 0           \\
        0            & 0 & 1 & 0           \\
        -\beta\gamma & 0 & 0 & \gamma      \\
      \end{pmatrix}
      \smref{5.2.46}
    \end{equation*}
    Für die Rücktransformation gilt:
    \begin{equation*}
      x^\mu = \qty(\Lambda^{-1})^\mu_{\ \nu} x'^\nu 
      \text{ mit } 
      \Lambda^\mu_{\ \nu}=
      \begin{pmatrix}
        \gamma       & 0 & 0 & \beta\gamma\\
        0            & 1 & 0 & 0           \\
        0            & 0 & 1 & 0           \\
        \beta\gamma  & 0 & 0 & \gamma      \\
      \end{pmatrix}
      \smref{5.2.48}
    \end{equation*}

    Man erinnere sich daran dass:
      $(\Lambda^{-1})^\mu_{\ \nu}
      =g_{\mu\sigma}\Lambda^{\sigma}_{\ \gamma}g^{\gamma\nu}
      =\Lambda_{\mu}^{\ \nu}
      \smref{5.2.25}$

  \item Man schreibe den kontravarianten Vierer-Vektor für ein %5
    Raum-Zeit-Ereignis x. Wie transformiert sich ein kontravarianter
    Lorentz-Vierer-Vektor?

    \begin{equation*}
      x^\mu =
      (ct, x, y, z)^T
      \smref{5.2.3}
    \end{equation*}
    \begin{equation*}
    \qquad x'^\mu = \Lambda^\mu_{\ \nu} x^\nu 
    \qquad x^\mu = (\Lambda^{-1})^\mu_{\ \nu} x'^\nu 
    \smref{5.2.12}
    \end{equation*}

  \item Wie erhält man den kovarianten Vierer-Vektor für x mit dem %6
    metrischen Tensor?

    \begin{equation*}
      x_\mu = g_{\mu\nu}x^\nu=
      (ct, -x, -y, -z)^T
      \smref{5.2.8}
    \end{equation*}

  \item Wie sieht das invariante Skalarprodukt zwischen zwei %7
    Vierer-Vektoren in einem Koordinaten-System aus, ohne und mit
    dem metrischen Tensor geschrieben?
    
    Seien $v^\mu$, $w^\mu$ Vierervektoren der Form: 
    $(x_0, x_1, x_2, x_3)^T$ mit $x=v,w$

    \begin{equation*}
      \begin{split}
      \langle v^\mu , w^\mu \rangle 
      &\coloneqq v^\mu w_\mu = 
      v^\mu g_{\mu\nu} w^\nu
      =
      \begin{pmatrix}
        v_0 \\ v_1 \\ v_2 \\ v_3
      \end{pmatrix}
      \begin{pmatrix}
        1 & 0  & 0  &  0\\
        0 & -1 & 0  &  0\\
        0 & 0  & -1 &  0\\
        0 & 0  & 0  & -1\\
      \end{pmatrix}
      \begin{pmatrix}
        w_0 \\ w_1 \\ w_2 \\ w_3
      \end{pmatrix}\\
      &=v_0w_0 - (v_1w_1 + v_2w_2 + v_3w_3)
      \smref{5.2.6}
      \end{split}
    \end{equation*}

    Man bemerke dass $w_\mu=(w_0, -w_1, -w_2, -w_3)^T$

    Für $x_0=ct_x$ und $x_{i\in\{1,2,3\}}$ (also im Koordinatensystem) 
    folgt natürlich:

    \begin{equation*}
      v^\mu w_\mu = c^2(t_vt_w) - (v_xw_x + v_yw_y + v_zw_z)
    \end{equation*}

  \item Warum ist die Gescwhindigkeit difineiert als $v^\mu=\dv t x^\mu$ %8
    kein Lorentz-Vektor? Welche verwandtete Geschwindigkeit ist ein
    Lorentz-Vektor?

    $v^\mu$ ist kein Vierer-Vektor weil es nicht sich nicht
    wie ein Lorentz-Vektor. Das heißt, dass:
    \begin{equation*}
      \dv {t'} x'^\mu \neq \Lambda^{\mu}_{\ \nu}\dv t x^\nu
      \smref{5.2.61}
    \end{equation*}
    Dies ist einfach zu Zeigen. Es gilt zwar: 
    $v^\mu=\dv t x^\mu=(c,\vb* v)$. Es gilt nun:
    \begin{equation*}
      v_\mu v^\mu = c^2-\vb* v^2
    \end{equation*}
    Was offensichtlich kein Lorentz-Vektor entspricht.

    $u^\mu$, die \textbf{Weltgeschwindigkeit}, 
    ist schon ein Lorentz-Vektor,
    mit $u^\mu=\dv \tau x^\mu$ und
    $$d\tau=\frac{1}{c}ds=\gamma^{-1}dt \smref{5.2.54\,\&\,55}$$

  \item Wie lauten die Komponenten des Vierer-Potenitals und des %9
    Vierer-Stroms in einem gegebenen Koordinatensystem? Wie schreibt
    sich die Wellengleichung damit?

    \begin{equation*}
      A^\mu(x^\alpha)=
      \begin{pmatrix}
        \Phi\\ A_x\\ A_y\\ A_z
      \end{pmatrix}
      =
      \begin{pmatrix}
        \Phi(x^\alpha) \\ \vb* A(x^\alpha)
      \end{pmatrix}
      \smref{5.3.8}
    \end{equation*}
    \begin{equation*}
      j^\mu(x^\alpha) =
      \begin{pmatrix}
        c\rho\\ j_x\\ j_y\\ j_z
      \end{pmatrix}
      =
      \begin{pmatrix}
        c\rho(x^\alpha) \\ \vb* j(x^\alpha)
      \end{pmatrix}
      \smref{5.3.1}
    \end{equation*}

    Alternativ gilt auch $j^\mu=\rho \dv{x^\mu}{t}
    =\frac{\rho}{\gamma}u^\mu \smref{5.3.2\ \&\,4}$

    Sei 
    $\square
    = -\qty(\frac{1}{c^2}\partial_t^2-\Delta)
    = -\qty(\pdv[2]{x_0}-\pdv[2]{x_1}-\pdv[2]{x_2}-\pdv[2]{x_3})
    = -\partial_\mu\partial^\mu \smref{5.3.11}$
    
    Dann gilt für die Wellengleichung von $A^\mu$:
    \begin{equation*}
      \square A^\mu = -\frac{4\pi}{c}j^\mu 
      \quad\text{bzw.}\quad
      \partial_\nu\partial^\nu A^\mu = \frac{4\pi}{c}j^\mu 
      \smref{5.3.12\,\&\,13}
    \end{equation*}

    \textbf{Notiz:} $\pdv{x^\mu}=\partial^\mu=(\frac{1}{c}\dv t, \nabla)$

  \item Man gebe den Feldstärke-Tensor als Funktion des %10
    Vierer-Potentials
    und Matrix mit den Komponenten von $\vb E$ und $\vb B$ an.

    \begin{equation*}
      F^{\mu\nu}=\partial^\mu A^\nu - \partial^\nu A^\mu=
      \begin{pmatrix}
        0   & -E_x & -E_y & -E_z\\ 
        E_x & 0    & -B_z & B_y \\ 
        E_y & B_z  & 0    & -B_x\\ 
        E_z & -B_y & B_x  & 0   \\ 
      \end{pmatrix}
    \smref{5.3.16}
    \end{equation*}

  \clearpage
  \item Wie lauten die homogenen Maxwell-Gleichungen in forminvarianter %11
    Schreibweise?

    \begin{equation*}
      \partial^\mu F^{\nu\lambda} + 
      \partial^\lambda F^{\mu\nu} + 
      \partial^\nu F^{\lambda\mu}
      =0 \smref{5.3.23}
    \end{equation*}
    Oder auch:
    \begin{equation*}
      \partial_\mu \tilde F^{\mu\nu}
      =0 \smref{5.3.28}
    \end{equation*}

    Dabei ist $\tilde F^{\mu\nu}=\frac{1}{2}
    \epsilon^{\mu\nu\eta\gamma}F_{\eta\gamma}$
    
    Mit \sref{5.3.25}
    \begin{equation*}
      \epsilon^{\mu\nu\eta\gamma}=
      \begin{cases}
        1&\text{gerade Permutation}\\
        -1&\text{ungerade Permutation}\\
          0&\text{nicht alle verschieden}
      \end{cases}
    \end{equation*}

    Es folgt daraus direkt dass
    \begin{equation*}
      \begin{array}{lll}
        \div B = 0 & (\nu = 0) & \smref{5.3.27} \\
        \curl E = -\frac{1}{c}\partial_t B & (\nu = 1,2,3) 
                                           & \smref{5.3.27} 
      \end{array}
    \end{equation*}

  \item Wie lauten die inhomogenen Maxwell-Gleichungen in %12
    forminvarianter Schreibweise?

    Wegen $\partial_\mu A^\mu=0 \smref{5.3.15}$ 
    (folgt aus Kontinuitätsgleichung), gilt auch
    
    \begin{equation*}
      \partial_\mu F^{\mu\nu}
      =\partial_\mu\qty[\partial^\mu A^\nu-\partial^\nu A^\mu]
      =\partial_\mu\partial^\mu A^\nu - \partial^\nu 
      \underbrace{\partial_\mu A^\mu}_{0}
      =-\square A^\nu = \frac{4\pi}{c}j^\nu \smref{5.3.20} 
    \end{equation*}
    Es folgt daraus direkt dass
    \begin{equation*}
      \begin{array}{lll}
        \div \vb E = 4\pi \rho &(\nu=0) &\smref{5.3.21}\\
        \curl \vb B = \frac{4\pi}{c}\vb* j + \frac{1}{c}\partial_t \vb E
                    &(\nu=1,2,3) &\smref{5.3.22}
      \end{array}
    \end{equation*}

  \item Wie schreibt sich die Lorentz-Kraft für eine Ladung q auf %13
     forminvariante Weise?

    \begin{equation*}
      K^\mu=\dv \tau p^\mu =m \dv \tau u^\mu = \frac{q}{c} F^{\mu\nu}u_\nu
      \smref{5.4.53}
    \end{equation*}

  \clearpage
  \item Man gebe das kirchhoff'sche Beugungsintegral [für %14
    Lichteinfall] auf eine Blendenöffnung bei weit 
    entfernter Lichtquelle und Senkrechtem Einfall an.
    
    Sei $G$ die Green'sche Funktion zur Helmholzgleichung mit
    \begin{equation*}
      G(\vb* r, \vb* r')
      =-\frac{1}{4\pi}\frac{e^{ik\abs{\vb* r -\vb* r'}}}{\abs{\vb* r - \vb* r'}}
    \end{equation*}
    Sei weiter $\psi(\vb* r)$ eine skalare Funktion die die Helmholzgleichung
    Erfüllt (z.B.:\ Wellenamplitude).
    So ergibt sich für das allgemeine Kirchoff Integral
    \begin{equation*}
      \psi(\vb* r')=\int_{\partial V} d\vb* A 
      \cdot 
      \qty[\psi(\vb* r)\grad G(\vb* r, \vb* r')-
      G(\vb* r, \vb* r')\grad\psi(\vb* r)]
      \smref{6.1.6}
    \end{equation*}
    Wobei $\partial V$ als die Blendenöffnung $A$ gesehen werden soll.

    Sei nun die Abstand Quelle-Blende $r_Q$ viel größer als die 
    Breite der Blendenöffnung $a$ und die Abstand zur in der Mitte 
    der Blende gelegte Usprung $r$.\ D.h.: 
    $r\le a \ll r_Q$

    So ist 
    $k\abs{\vb* r -\vb* r_Q}\approx kr_Q\qty(1-\frac{1}{2}\frac{r^2}{r^2_Q})$

    Sei nun $ka^2/r_Q\ll \pi$ bzw $\lambda \gg 2a^2/r_Q$ so ist
    \begin{equation*}
      \psi(\vb* r)=C\int_A dA \rr{e^{ik\abs{\vb* r-\vb* r'}}}\smref{6.1.14}
    \end{equation*}
    Dies ist bei Weit entfernter Lichtquelle der Falls.

  \item Was ergibt sich in der Fraunhofer-Näherung? Wann gilt diese? %15

    Die Fraunhofer-Näherung gillt für
    \begin{equation*}
      \frac{a}{d} \ll \frac{\lambda}{a} \smref{6.2.1}
    \end{equation*}
    Mit $a$ die Breite der Öffnung, $d$ die Abstand zur Schirm und
    $\lambda$ die Wellenlänge des Lichtes.

    Die Fraunhofer-Näherung ergibt für die (komplexe) Amplitude der
    elektromagnetische Wellen am Schirm:
    \begin{equation*}
      \psi(x)\approx \frac{e^{ikd}}{d}\int_{-a/2}^{a/2}dx'
      e^{-i \frac{k}{d}xx'}
      =
      -a \frac{e^{ikd}}{d} \frac{\sin(qx)}{qx} 
      \smref{6.2.6}
    \end{equation*}
    Mit $q=\frac{k}{d} \frac{a}{2}$ und wobei $x$ die Schirmkoordinate 
    darstellt.
    Dabei beobachtet man die Intensität $I=\abs{\psi(x)}^2$ auf dem Schirm.

\end{enumerate}
