\begin{center}
  \begin{tabular}{ll|ll|ll}
    \toprule
    Frage & Sicherheitsgrad & Frage & Sicherheitsgrad 
          & Frage & Sicherheitsgrad \\
    \midrule
    1&                  & 6 &            & 11&       \\ 
    2&                  & 7 &            & 12&       \\
    3&                  & 8 &            & 13&       \\
    4&                  & 9 &            & 14&       \\
    5&                  & 10&            & 15&       \\
    \bottomrule
  \end{tabular}
\end{center}

\begin{enumerate}
  \item Was bedeutet das spezielle Relativitätsprinzip für physikalische
    Gesetze und die Lichtgeschwindigkeit?

    Die Lichtgeschwindigkeit ist eine Konstante in jedem Bezugssystem.
    Die physikalische Gesetze sollen in jedes Bezugssytem gleich sein,
    unterschiedliche Observierer können aber unterschiedliche Gründen
    angeben weshalb ein Erreignis statt findet.

  \item Wie lautet die Formel für den invarianten Raum-Zeit-Abstand ds in
    einem Koordinatensystem? Warum folgt aus der Invarianz von ds die
    Invarianz der Lichgeschwindigkeit.
    
    \begin{equation*}
      ds^2 = c^2 t^2 - \vb x^2
    \end{equation*}

  \item Welche Bedingungen stellt man an eigentliche orthochrone
    Lorentz-transformationen?

    %TODO

  \item Wie lautet die Lorentz-Transformation für einen Boost in 
    $z$-Richtung? Wie sieht die Inverse aus?
    
    Sei $S$ ein Ruhesystem und bewege $S'$ sich mit relativen 
    Geschwindigkeit $v$ in $z$-Richtung. Man definiere die (einheitslosen)
    Größen:
    $\beta = \frac{v}{c}$, $\gamma=\qty(1-\frac{v^2}{c^2})^{-\frac{1}{2}}$
    
    Für die Hintransformation gilt:
    \begin{equation*}
      x'^\mu = \Lambda^\mu_{\ \nu} x^\nu 
      \text{ mit } 
      \Lambda^\mu_{\ \nu}=
      \begin{pmatrix}
        \gamma       & 0 & 0 & -\beta\gamma\\
        0            & 1 & 0 & 0           \\
        0            & 0 & 1 & 0           \\
        -\beta\gamma & 0 & 0 & \gamma      \\
      \end{pmatrix}
    \end{equation*}
    Für die Rücktransformation gilt:
    \begin{equation*}
      x^\mu = \qty(\Lambda^{-1})^\mu_{\ \nu} x'^\nu 
      \text{ mit } 
      \Lambda^\mu_{\ \nu}=
      \begin{pmatrix}
        \gamma       & 0 & 0 & \beta\gamma\\
        0            & 1 & 0 & 0           \\
        0            & 0 & 1 & 0           \\
        \beta\gamma  & 0 & 0 & \gamma      \\
      \end{pmatrix}
    \end{equation*}

    Man erinnere sich daran dass:
      $(\Lambda^{-1})^\mu_{\ \nu}
      =g_{\mu\sigma}\Lambda^{\sigma}_{\ \gamma}g_{\gamma\nu}$

    \item Man schreibe den kontravarianten Vierer-Vektor für ein
      Raum-Zeit-Ereignis x. Wie transformiert sich ein kontravarianter
      Lorentz-Vierer-Vektor?

      \begin{equation*}
        x^\mu =
        \begin{pmatrix}
          ct\\ x \\ y \\ z
        \end{pmatrix}
      \qquad x'^\mu = \Lambda^\mu_{\ \nu} x^\nu 
      \qquad x^\mu = (\Lambda^{-1})^\mu_{\ \nu} x'^\nu 
      \end{equation*}

    \item Wie erhält man den kovarianten Vierer-Vektor für x mit dem
      metrischen Tensor?

      \begin{equation*}
        x_\mu = g_{\mu\nu}x^\nu
        \begin{pmatrix}
          ct\\ -x \\ -y \\ -z
        \end{pmatrix}
      \end{equation*}

    \item Wie sieht das invariante Skalarprodukt zwischen zwei 
      Vierer-Vektoren in einem Koordinaten-System aus, ohne und mit
      dem metrischen Tensor geschrieben?
      
      Seien $v^\mu$, $w^\mu$ Vierervektoren der Form: 
      $(x_0, x_1, x_2, x_3)^T$ mit $x=v,w$

      \begin{equation*}
        \begin{split}
        \langle v^\mu , w^\mu \rangle 
        &\coloneqq v^\mu w_\mu = 
        v^\mu g_{\mu\nu} w^\nu
        =
        \begin{pmatrix}
          v_0 \\ v_1 \\ v_2 \\ v_3
        \end{pmatrix}
        \begin{pmatrix}
          1 & 0  & 0  &  0\\
          0 & -1 & 0  &  0\\
          0 & 0  & -1 &  0\\
          0 & 0  & 0  & -1\\
        \end{pmatrix}
        \begin{pmatrix}
          w_0 \\ w_1 \\ w_2 \\ w_3
        \end{pmatrix}\\
        &=v_0w_0 - (v_1w_1 + v_2w_2 + v_3w_3)
        \end{split}
      \end{equation*}

      Man bemerke dass $w_\mu=(w_0, -w_1, -w_2, -w_3)^T$

    \item Warum ist die Gescwhindigkeit difineiert als $v^\mu=\dv t x^\mu$
      kein Lorentz-Vektor? Welche verwandtete Geschwindigkeit ist ein
      Lorentz-vektor?

      $v^\mu$ ist kein Vierer-Vektor weil $\ldots$

      $u^\mu$ ist ein Vierer-Vektor, mit $u^\mu=\dv \tau x^\mu$

    \item Wie lauten die Komponenten des Vierer-Potenitals und des
      Vierer-Stroms in einem gegebenen Koordinatensystem? Wie schreibt
      sich die Wellengleichung damit?

      \begin{equation*}
        A^\mu=
        \begin{pmatrix}
          \Phi(x^\alpha) \\ A_x(x^\alpha) \\ A_y(x^\alpha) \\ A_z(x^\alpha)
        \end{pmatrix}
        =
        \begin{pmatrix}
          \Phi(x^\alpha) \\ \vb A(x^\alpha)
        \end{pmatrix}
        \qquad
        j^\mu =
        \begin{pmatrix}
          \rho(x^\alpha) \\ j_x(x^\alpha) \\ j_y(x^\alpha) \\ j_z(x^\alpha)
        \end{pmatrix}
        =
        \begin{pmatrix}
          \rho(x^\alpha) \\ \vb j(x^\alpha)
        \end{pmatrix}
      \end{equation*}

      TODO: Wellengleichung

    \item Man gebe den Feldstärke-Tensor als Funktion des Vierer-Potentials
      und Matrix mit den Komponenten von $\vb E$ und $\vb B$ an.

      \begin{equation*}
        F^{\mu\nu}=\partial^\mu A^\nu - \partial^\nu A^\mu=
        \begin{pmatrix}
          0   & -E_x & -E_y & -E_z\\ 
          E_x & 0    & -B_z & B_y \\ 
          E_y & B_z  & 0    & -B_x\\ 
          E_z & -B_y & B_x  & 0   \\ 
        \end{pmatrix}
      \smref{5.3.16}
      \end{equation*}

    \item Wie lauten die homogenen Maxwell-Gleichungen in forminvarianter
      Schreibweise?

    \item Wie lauten die inhomogenen Maxwell-Gleichungen in forminvarianter
      Schreibweise?

    \item Wie schreibt sich die Lorentz-Kraft für eine Ladung q auf
      forminvariante Weise?

    \item Man gebe das kirchhoff'sche Beugungsintegral [für
      Lichteinfall] auf eine Blendenöffnung bei weit 
      entfernter Lichtquelle und Senkrechtem Einfall an.

    \item Was ergibt sich in der Fraunhofer-Näherung? Wann gilt diese?

\end{enumerate}
