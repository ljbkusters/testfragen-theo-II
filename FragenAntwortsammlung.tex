\documentclass{scrartcl}
\usepackage[margin=2.5cm]{geometry} 
\usepackage[utf8]{inputenc}
\usepackage[ngerman]{babel}
\usepackage{physics}
\usepackage{hyperref}
\usepackage{booktabs}
\usepackage{bbold}
\usepackage{bm}
\usepackage{graphicx}
\usepackage{mathtools}
\usepackage{amsfonts}

\renewcommand{\thefootnote}{\fnsymbol{footnote}}

\newcommand{\rr}[1]{\frac{#1}{\abs{\vb{r}-\vb{r}'}}} 
\newcommand{\ds}{\displaystyle} 

% reference skript pages
\newcommand{\sref}[1]{(Skript #1)} 
\newcommand{\smref}[1]{\quad\text{(Skript #1)}} 

\newcommand{\subdir}{subtex}

\title{Fragen und Lösungen zur Vorlesung Theo II}
\author{Luc Kusters}
\date{WS 19/20}

\begin{document}

\maketitle

\noindent
Die Lösungen die hier zusammengefasst wurden, kommen von mir und
aus die Theo II Whatsappgruppe. 
\textbf{Weder absolute Vollständichkeit noch absolute Korrektheit wird garrantiert!} Aber wo ich mir unsicher war habe ich Sachen nachgefragt, oder 
mittels dem Skript geklärt.
Bei Anmerkungen, Fragen, Korrekturen oder sonstiges, könnt ihr mich gerne 
per email erreichen:\\ 
ljbkusters@gmail.com \\

\noindent
Vielen Dank an jedem, der mitgeholfen hat! Insbesondere:
\begin{itemize}
  \item Prof. Honerkamp / Dr. Mück für die Fragen
  \item ``Der Andere'' für Seine Bilder mit Lösungen in der Whatsappgruppe
  \item Lais für seine Aufmerksamkeit bei Fragen 2.7 und 2.8, wo immer noch Konstanten falsch definiert waren.
  \item ``JP'' für Seine Aufmerkungen zur Fragen 1.8
\end{itemize}

\clearpage
\newpage
\setcounter{page}{1}
\tableofcontents
\newpage

\section{Grundkentnisse Elektrostatik I}
\begin{center}
  \begin{tabular}{ll|ll|ll}
    \toprule
    Frage & Sicherheitsgrad & Frage & Sicherheitsgrad 
          & Frage & Sicherheitsgrad \\
    \midrule
    1 & Sehr sicher & 6  & Sicher      & 11 & Sehr sicher \\ 
    2 & Sehr sicher & 7  & Sehr sicher & 12 & Sehr sicher \\
    3 & Sehr sicher & 8  & Sehr sicher & 13 & Sehr sicher \\
    4 & Sehr sicher & 9  & Sehr sicher & 14 & Sehr sicher \\
    5 & Sehr sicher & 10 & Sicher      & 15 & Sicher\\
    \bottomrule
  \end{tabular}
\end{center}
\begin{enumerate}
  \item Man berechne $\ds \grad\rr{1}$.
        $$\ds \grad\rr{1}=
         -\frac{\vb r -\vb r'}{\abs{\vb r - \vb r'}^3}\quad
         \qty(=-\grad'\rr{1})\quad\text{(Skript 1.3.1)}$$ 

  \item Was ergibt $\ds \Delta\rr{1}$?
        $$\ds \Delta \rr{1} = -4\pi \delta(\vb r-\vb r')
        \quad\text{(Skript 1.3.23)}$$
    
  \item Wie lauten die beiden Feldgleichungen für das 
        elektrische Feld in der Elektrostatik?
        $$\ds \div \vb E=4\pi k\rho(\vb r)\,\qquad\curl\vb E=0
         \qquad\qty(\text{SI:}\,k=\frac{1}{4\pi\epsilon_0})$$
        \begin{center}
          (Skript 1.3.25 und 1.3.6)
        \end{center}
  
  \item Wie lautet die Bestimmungsgleichung für das elektrostatische 
        Potential bei gegebener Ladungsdichte $\rho(\vb r)$?
        $$\phi(\vb r)=k\int d^3r'\rr{\rho(\vb r')}
        \quad\text{(Skript 1.3.3)}$$

  \item Wie berechnet sich das elektrische Feld $\vb E (\vb r)$ aus dem
        Potential $\Phi(\vb r)$?
        $$\ds \vb E(\vb r) = - \grad \Phi (\vb r)
        \quad\text{(Skript 1.3.2)}$$

  \item Wie Berechnet man den Potentialunterschied \textbf{[Spannung]}
        zwischen den Orten 
        $\vb r_2$ und $\vb r_1$ bei gegebenem Feld $\vb E(\vb r)$?
        $$\ds U(\vb r_1, \vb r_2)=U_2-U_1
         =\int_{\vb r_1}^{\vb r_2} \vb E(\vb r)\cdot d\vb s
         =-\int_{\vb r_1}^{\vb r_2} \vb \grad \Phi (\vb r)\cdot d\vb s
         =-\qty(\Phi(\vb r_2)-\Phi(\vb r_1))
         \quad\text{(Skript 3.2.6)}$$
        \textbf{Notiz}: Wegen $\vb E=-\grad\Phi$ ist $\vb E$ ein reines
        Gradientenfeld. Dies heißt, daß das Integral über $\vb E$
        wegunabhängig ist (nur gültig in der Elektrostatik!).

  \item Was besagt der satz von Gauß als mathematische Aussage?
        $$\ds \int_V d^3r \qty[\div \vb F(\vb r)]=
         \oint_{\partial V}d\vb A\cdot \vb F(\vb r)
         \quad\text{(Skript 1.3.12)}$$\\
        \textbf{Notiz:} In 3 Dimensionen, wobei $V$ ein Volumen darstellt,
        und $\partial V$ dessen Randfläche darstellt.

  \item Welche Aussage erhält man mit dem Satz von Gauß in der 
        Elektrostatik in Anwesenheit von Ladungen in einem Volumen V?
        \begin{equation*}
          \int_V dV \div \vb E(\vb r) 
          = \int_{\partial V} d\vb A \cdot \vb E(\vb r) 
          = 4\pi k \int_V \rho(\vb r)
          = 4\pi k Q_{\text{ges}}
        \end{equation*}
        Es folgt damit für einfach zusammenhängenden Volumina:
        $$\vb E(\vb r)\eval_{\abs{\vb r} > R}
        \propto Q_{\text{ges}} \text{in} V_R
          \quad\text{(Skript 1.3.33)}$$\\
        \textbf{Oder in Wörter:} Das Elektrische Feld außerhalb 
        einem Volumen 
        $V_R$ mit maximale Ausdehnung $R$
        ist proportional zur Gesammtladung $Q_{\text{ges}}$ 
        im Volumen $V_R$. Beispiel: Homogen Geladene Kugel.

  \item Was besagt der Satz von Stokes Mathematisch?
        $$\ds \int_A d\vb A \cdot \qty[\curl \vb F(\vb r)]=
         \oint_{\partial A}d\vb s\cdot \vb F(\vb r)
         \quad\text{(Skript 1.4.10)}$$
        \textbf{Notiz:} In 3 Dimensionen, wobei $A$ eine Fläche 
        darstellt,
        und $\partial A$ dessen Randkurve darstellt.

  \item Wann ist die Lösung der Poisson-Gleichung eindeutig?
        \begin{center}
          Falls die Randbedingungen vorgegeben sind. (Skript §1.5)
        \end{center}
        \textbf{Notiz:} $G(\vb r, \vb r')=\rr{1} + F(\vb r, \vb r')$ 
        ist wegen $F(\vb r, \vb r')$ nicht eindeutig!
        
  \item Man gebe das infinitesimale Flächenelement auf der Oberfläche 
        einer Kugel mit Radius R an.
        $$\ds dA = R^2\sin\theta d\theta d\phi
        \quad d\vb A = \vu e_r dA \smref{1.3.25}$$
        \textbf{Notiz:} $\ds\int_{\partial B_R(0)} dA=\int\delta(r-R)dV
         =\int \delta(r-R)r^2dr\sin\theta d\theta d\phi
         =R^2 \int \sin\theta d\theta d\phi$

  \item Man berechne das Volumen einer Kugel mit Radius R 
        in Kugelkoordinaten\\
        $$\ds V_k=\int_{\sqrt{x^2+y^2+z^2}\le R} 1\cdot dxdydz
         =\int_0^{2\pi}\int_0^\pi\int_0^R 1\cdot r^2dr 
          \sin\theta d\theta d\phi$$
        $$\ds V_k=\int_0^R r^2 dr \int_{-1}^{1} d\cos\theta 
                 \int_0^{2\pi}d\phi
         =\qty[\frac{1}{3}R^3]\qty[2]\qty[2\pi]=\frac{4\pi}{3}R^3$$
        \begin{center}
          (Skript 1.3.36$\sim$38)
        \end{center}

  \item Wie sieht die Enwicklung einer Funktkion $f(\theta,\phi)$ in
        Kugelflächenfunkionen aus und wie bestimmt man die 
        Entwicklungskoffizienten?
        $$\ds f(\theta, \phi)=\sum_{lm} f_{lm}Y_{lm}(\theta,\phi)
        \qquad l\in \mathbb{N}_0;\quad \abs{m}\le l; 
        \quad m\in\mathbb{Z}$$
        $$\ds f_{lm}=const.
         =\int_{-1}^{1} d\cos\theta\int_0^{2\pi}d\phi 
         Y^*_{lm}(\theta,\phi)f(\theta,\phi)
         =\int d\Omega Y^*_{lm}(\Omega)f(\Omega)$$
         \textbf{Notiz:} Falls man $f(\theta,\phi)$ als linear kombination
         von $Y_{lm}(\theta,\phi)$ schreiben kann, kann man $f_{lm}$
         sehr einfach bestimmen (es gilt zwar $\ds \int d\Omega 
         Y_{lm}(\Omega)Y_{l'm'}^*(\Omega)=\delta_{ll'}\delta_{mm'}$)
        \begin{center}
          (Skript 1.5.70$\sim$71)
        \end{center}

  \item Wie lautet der Ansatz für die Lösung der Laplace-Gleichung in
        Kugelfächenfunktionen?\\
        $$f(r,\theta,\phi)=
        \sum_{lm}\qty(a_{lm}r^l+b_{lm}r^{-(l+1)})Y_{lm}(\theta,\phi)
        \quad\text{(Skript 1.5.80)}$$
         
  \item Welche zwei wichtigen Gleichungen erfüllt die Dirichlet-Green'sche
        Funktion $G_D(\vb r, \vb r')$?

        Allgemein Wichtige Bedingungen der Greenshce Funktion
        \begin{center}
          \begin{enumerate}
            \item $\Delta G(\vb r, \vb r') = -4\pi\delta(\vb r -\vb r')
              \smref{1.5.90}$
            \item Sei $\Delta g(x) = f(x)$ eine Lösung der inhomogenen  
              Laplace Gleichung, so ist 
              $$ g(x)=\Delta\int G(x,x')f(x')dx'$$
            \item $G(\vb r, \vb r') \to 0$ bzw. $F(\vb r, \vb r')\to 0$ 
              für $\vb r'\to\infty
              \smref{§1.5.6, nach Formel 1.5.94}$
          \end{enumerate}
        \end{center}
        \textbf{Notiz:} Relation (b) heißt in Wörter, das man die gesuchte
        Funktion $g(x)$, welche die Poissong-Gleichung erfüllt, finden
        kann, indem man das Integral berechnet. Für uns würde bei g(x)
        Durchaus $\Phi$ stehen, und bei $f(x)$ durchaus $\rho$.
        Mit diese Relation kann man das 
        Poisson-Problem
        von ein DGL in ein Integral umwandern, falls man $G(x,x')$ kennt.
        Dies ist der Hauptpunkt weshalb man gerne Greensche Funktionen
        Benützt, denn Integralen sind durchaus einfacher 
        zu lösen als DGLs (vor allem in der Praxis, wobei numersich
        lösen möglich ist).

        \textbf{Bei Dirichlet-RWB.} (Hier gefragt):
        \begin{center}
        \begin{enumerate}
          \item $\ds G_D(\vb r, \vb r')= G_D(\vb r', \vb r)
            \smref{1.5.99}$ 
          (für reine Dirichlet Randbedingungen) (Skript §1.5.6)\\
        \item $\ds \Phi(\vb r)=k\int_V d^3r' G_D(\vb r, \vb r')\rho(\vb r)
           -\frac{1}{4\pi}\int_{\partial V}d\vb A'\cdot\phi(r')
           \grad' G_D(\vb r, \vb r')$ (Skript 1.5.90)\\
        \end{enumerate}
        \end{center}
        Bei von Neumann-RWB.:
        \begin{enumerate}
          \item $\ds \Phi(\vb r')=
            k\int_V d^3r G_N(\vb r,\vb r')\rho(\vb r)
            +\frac{1}{4\pi} \int_{\partial V} d\vb A \cdot 
            G_N(\vb r,\vb r)\grad\phi(r) + C$
        \end{enumerate}
\end{enumerate}

\newpage
\section{Grundkenntinsse Elektrostatik II / Magnetostatik}
\begin{center}
  \begin{tabular}{ll|ll|ll}
   \toprule
   Frage & Sicherheitsgrad & Frage & Sicherheitsgrad 
         & Frage & Sicherheitsgrad \\
   \midrule
   1& Sehr sicher      & 6 & Sicher           & 11& Sehr sicher\\ 
   2& Sehr sicher      & 7 & Sehr sicher      & 12& Sehr sicher\\
   3& Sehr sicher      & 8 & Sehr sicher      & 13& Sehr sicher\\
   4& Sehr sicher      & 9 & Sicher           & 14& Sicher\\
   5& Sicher           & 10& Sicher           & 15& Sicher\\
   \bottomrule
  \end{tabular}
\end{center}
\begin{enumerate}
  \item Wie lautet die Taylor-Entwicklung von $\rr{1}$ in den Komponenten
        von $\vb* r'$ bei $\vb* r'=0$ bis einschließlich der Zweiten
        Ordnung?
        $$ 
        \rr{1}\approx \frac{1}{r}+\frac{\vb* r\cdot\vb* r'}{r^3} +
        \frac{1}{2}\sum_{i,j=1}^3\frac{3r_{ij}-\delta_{ij}r^2}{r^5}
        r_i'r_j'
        \quad\text{(Skript 1.6.2)}$$
        \textbf{Notiz 1:} Dies ist eine ``Fernfeld'' Näherung, d.h. 
        eine Gute Näherung für $r\gg 0$ (denn für $r=0$ divergiert
        die Näherung).\\
        \textbf{Notiz 2:} Für genauere Information zur mehrdimensionale
        Taylor-Entwicklung, schaue im HöMa II Skript oder schaue hier 
        nach:~\url{https://de.wikipedia.org/wiki/Taylorreihe#Mehrdimensionale_Taylorreihe}

  \item Man drücke das elektrostatische Potential $\Phi(\vb* r)$ einer
        räumlich begrenzte Ladungsverteilung $\rho(\vb* r')$ mittels
        Monopol, den Dipolvektor und den Quadrupoltensor aus.
        $$
        \Phi(\vb* r)\approx k\frac{Q_\text{ges}}{r} +
        k\frac{\vb* r\cdot \vb* p}{r^3} +
        \frac{k}{2} \sum_{ij}\frac{r_ir_j}{r^5}Q_{ij}
        \smref{1.6.6}$$

  \item Man gebe Formel für das Monopolmoment, den Dipolvektor und den
        Quadrupoltensor an.
        \begin{center}
        \begin{tabular}{lll}
          Monopol:    & $\ds Q_{\text{ges}}=\int d^3r\rho(\vb* r)$
                        &\sref{1.6.7}\\
          Dipol:      & $\ds p_i=\int d^3r\rho(\vb* r)r_i
                        \quad\vb* p=p_i\vu{e}_i $
                        &\sref{1.6.8}\\
          Quadrupol:  & $\ds Q_{ij}=\int d^3r\rho(\vb* r)
          \qty(3r_ir_j - \delta_{ij}\abs{\vb* r}^2)$
                        &\sref{1.6.9}\\
        \end{tabular}
        \end{center}
        \textbf{Notiz 1:} $Q_{ij}$ hat nur 5 Freiheitsgraden! D.h.\ man kann           mit nur 5 Rechnungen alle (9) Quadrupol Elemente berechnen! Es
        gilt zwar $Q_{ij}=Q_{ji}$ und 
        $\text{sp}(\bm Q)=\sum_i Q_{ii}=0$.\\
        \textbf{Notiz 2:} Die hier nicht gefragte Kugelkoordinaten
        Multipolentwicklung ist auch sehr wichtig! 
        (Siehe Skript §1.6.6 S.42)

  \item Wie koppeln Monopol und Dipol an ein externes elektrostatisches
        Potential bzw. Feld?
        
        Für eine Multipol im Zentriert um $\vb* r_0$:
        $$E_{\text{pot}}=\Phi(\vb* r_0)Q-\vb E(\vb* r_0)\cdot\vb* p 
        \qquad\qty(-\frac{1}{6}\sum_{ij}\pdv{x_j}E_i(\vb* r_0)Q_{ij}
        +\ldots)
        \smref{1.7.7}$$
        \textbf{Notiz:} Energie minimierung für $\vb* p\parallel \vb E$
        und gleich ausgerichtet (also $\vu e_{\vb* p}=\vu e_{\vb E}$).\\

  \clearpage
  \item Wie lautet die elektrostatische Energie einer Ladungsverteilung?
        Man gebe zwei Äquivalente Ausdrücke, einerseits mit $\rho(\vb* r)$
        und andersseits mit $\vb E(\vb* r)$.
        \begin{center}
        \begin{tabular}{lll}
          mit $\rho$    & $\ds E_{\text{WW}}=
                          \frac{1}{2}\int d^3 r\rho(\vb* r)
                          \Phi_\rho(\vb* r)$
                            &(Skript 1.7.9)\\
          mit $\vb E$   & $\ds E_{\text{WW}}=\frac{1}{8k\pi}
                           \int d^3 r\vb E^2(\vb* r)$
                            &(Skript 1.7.15)\\
        \end{tabular}
        \end{center}

  \item Wie ist die elektrische Suszeptibilität $\chi$ in einem 
        linearen, isotropen Medium definiert?
        \begin{center}
        \begin{tabular}{lll}
          Definition:             
          & $\vb P = \frac{1}{4\pi k} 
            \vu \chi(\vb E)\cdot \vb E\quad$
          & (Skript 1.8.1) \\
          Falls isotrop: 
          & $\vu\chi(\vb E)=
            \mathbb{1}\chi(\vb E)$
          & (Skript 1.8.2) \\
          Mit schwache abh.\ von $\vb E$:
          & $\chi(\vb E)\approx\chi
          \quad(\chi=const.\in\mathbb{R})$
            & (Skript 1.8.3) \\
          (lineares Medium)
        \end{tabular}
        \end{center}
        \textbf{Notiz 1:} $\vu \chi$ ist im allgemeinsten Fall 
        definiert als ein Matrix/Tensor.
        dies lässt zu, dass $\vu \chi$ in unterschiedliche Richtungen
        andere Auswirkungen hat (Polarisation kann z.B. in x-Richtung
        Stärker sein als in y-Richtung). Wir betrachten in diesen Kurs 
        aber im Allgemein
        lineare, isotrope Medien, wobei es keine Richtungsabhängigkeit
        gibt, und wobei $\chi$ auch nicht Stark abhängt von das äußere
        $E$-Feld. Dadurch kann
        man $\chi$ als eine Medienabhängige skalare Konstante 
        sehen/nähern.\\
        \textbf{Notiz 2:} Die im Skript gegebene Definition ist 
        \textbf{Falsch}! (aber richtig bis auf der Vorfaktor.)\\

  \item Wie lautet der Zusammenhang zwischen elektrischem Feld, 
        dielektrischer Verschiebung und Polarisation?
        $$\vb E= 4\pi k\vb D - 4\pi k\vb P \quad\text{(Skript 1.8.17)}$$
        \textbf{Notiz 1:} Die im Skript gegebene Definition ist 
        \textbf{Falsch}! (aber richtig bis auf die Vorfaktoren.)\\
        \textbf{Notiz 2:} Alternativ gilt $\vb D=\epsilon \vb E
        \stackrel{\text{(SI)}}=\epsilon_0\epsilon_r \vb E$ mit
        $\epsilon=\frac{1+\chi}{4\pi k}$ (Herleitung wie im Skript) und
        $\epsilon_r=1+\chi$
        (unter die richtige Bedingungen, siehe Skript) (Skript 1.8.17)\\

  \item Welche Feldgleichung erfüllt die dielektrische Verschiebung?
        $$\div \vb D=\rho_f(\vb* r)\quad\text{(Skript 1.8.4)}$$
        $$\qty(\div \vb P=-\rho_P(\vb* r))\quad\text{(Skript 1.8.6)}$$
        \textbf{Notiz:} Die im Skript gegebene Definition ist 
        \textbf{Falsch}! (aber richtig bis auf der Vorfaktor.)

  \item Was geschieht mit dem $1/r$-Potential von Punktladungen in
        Metallen?
        \begin{center}
          Die Punktladungen werden abgeschirmt (Potential fällt 
          exponentiell ab)
          (Skript 1.8.30)
        \end{center}
        \textbf{Notiz:} Siehe auch Yukawa Potential: 
        \url{https://en.wikipedia.org/wiki/Yukawa_potential} 

  \clearpage
  \textbf{[MAGNETOSTATIK]}
  \item Man schreibe die Kontinuitätsgleichung in differentieller Form
        und mit Integralen über ein Volumen $V$ bzw.\ dessen Oberfläche 
        $\partial V$.
        \begin{center}
        \begin{tabular}{lll}
          Differentieller Form 
                            & $\ds\pdv{t}\rho(\vb* r, t)=
                               -\div\vb* j(\vb* r, t)$
                              & (Skript 2.1.5)\\
          Integral Form     & $\ds\dot Q_V
                              =\int_V d^3r\partial_t\rho(\vb* r,t)=
                              -\int_{\partial V}d\vb* A\cdot
                              \vb* j(\vb* r, t)=-I_{\partial V}$
                              & (Skript 2.1.1 \& 3)\\
        \end{tabular}
        \end{center}
        \textbf{Notiz:} Bemerke das beim Integral form den Satz v. Gauß
        angwand wurde.

  \item Man schreibe Ladungsdichte $\rho(\vb* r)$ und Stromdichte
        $\vb* j(\vb* r)$ für $i=1,\ldots,N$ Punktladungen $q_i$ mit 
        Trajektorien $\vb* r_i(t)$.
        $$\rho(\vb* r)=\sum_i q_i\delta(\vb* r-\vb* r_i(t))
        \quad \text{(Skript 1.1.2)}$$
        $$\vb* j(\vb* r)=\sum_i q_i\dot(t){\vb* r}_i\delta(\vb* r-\vb* r_i(t))
        \quad \text{(Skript 2.1.2)}$$

  \item Wie lautet das Biot-Savart'sche Gesetz für $\vb* A(\vb* r)$ und
        $\vb B(\vb* r)$?
        $$\vb B(\vb* r)=k'\int d^3r'\frac{\vb* j(\vb* r')\times
        (\vb* r-\vb* r')}{\abs{\vb* r-\vb* r'}^3}
        \quad \text{(Skript 2.2.1)}$$
        $$\vb* A(\vb* r)=k'\int d^3r'\rr{\vb* j(\vb* r')}
        \quad \text{(Skript 2.3.2)}$$
        \textbf{Notiz 1:} $k'=\frac{\mu_0}{4\pi}$ (SI)\\
        \textbf{Notiz 2:} $\curl \vb* A(\vb* r) = \vb B(\vb* r)$ und 
        $\curl \rr{j(\vb* r')}
        =\frac{\vb* j(r') \times (\vb* r - \vb* r')}{\abs{\vb* r - \vb* r'}^3}$

  \item Wie lauten die zwei Feldgleichungen der Magnetostatik?
    \begin{equation*}
      \begin{array}{ll}
        \div \vb B(\vb* r) = 0&\smref{2.3.5}\\
        \curl \vb B(\vb* r)= 4\pi k'\vb* j(\vb* r)&\smref{2.3.12}
      \end{array}
    \end{equation*}

  \item Man berechne das magnetische Dipolmoment für eine Punktladung auf
        einer Kreisbahn mit Radius $R$ sowie Drehimpuls $\vb* L$.
        $$\vb* j(\vb* r,t)=q\vb* v(t)\delta(\vb* r-\vb* r(t))\smref{2.1.2}$$
        $$\vb* m(t) 
        = \frac{1}{2}\int d^3 r (\vb* r(t) \times \vb* j(\vb* r,t))
        =\frac{q}{2}\vb* r(t)\times\vb* v(t)=\frac{q}{2m}\vb* L(t)$$
        \textbf{Notiz:} Definition von $\vb* m$ \sref{2.4.8}

  \item Wie lauten die Zusammenhänge zwischen $\vb B$, $\vb H$ 
        und $\vb M$ in einem Paramagneten?

        Allgemein:
        $$\vb B = 4\pi k'\vb H + 4\pi k'\vb M\quad\text{(Skript 2.5.12)}$$
        $$\vb B = \mu_0 \vb H + \mu_0 \vb M \quad\text{(SI) 
        (Skript 2.5.17)}$$
        Im (Dia- und) Paramagneten:
        \begin{equation*}
            \vb M 
            = \chi_M \vb H 
            = \frac{1}{4\pi k'}\frac{\chi_M}{1+\chi_M} \vb B
        \end{equation*}
         \begin{equation*}
            \vb B = \mu \vb H \quad \text{mit}\quad
            \mu = 4\pi k'(1+\chi_M)
         \end{equation*}
          \begin{center}
            \sref{2.6.3$\sim$6}
          \end{center}

        \textbf{Notiz 1:} $\ds \curl \vb H = \vb* j_f$ und 
        $\ds \curl \vb M=\vb* j_M$\\
        \textbf{Notiz 2:} Die im Skript gegebene Definition ist 
        \textbf{Falsch}! (aber richtig bis auf der Vorfaktor.)
\end{enumerate}

\newpage
\section{Grundkenntinsse Elektrodynamik}
\setcounter{footnote}{0}
\begin{center}
  \begin{tabular}{ll|ll|ll}
    \toprule
    Frage & Sicherheitsgrad & Frage & Sicherheitsgrad 
          & Frage & Sicherheitsgrad \\
    \midrule
    1& Sehr sicher      & 6 & Sicher     & 11& Sicher\\ 
    2& Sehr sicher      & 7 & Sicher     & 12& Sicher\\
    3& Sicher           & 8 & Sicher     & 13& Sicher\\
    4& Sehr sicher      & 9 & Sicher     & 14& Sicher\\
    5& Sehr sicher      & 10& Sehr sicher& 15& Sicher\\
    \bottomrule
  \end{tabular}
\end{center}
\begin{enumerate}

  \item Wie lauten die vollen, makroskopischen Maxwell-Gleichungen für
        die Felder $\vb E(\vb* r, t)$ und $\vb B(\vb* r, t)$
        \begin{center}
        \scalebox{0.8}{
        \begin{tabular}{lll}
        Name          & Differentiell       
                            & Integral\\
        Gauß'scher Satz     
                      & $\ds \div \vb E(\vb* r, t) 
                        = 4\pi k\rho(\vb* r, t)$
                            & $\ds \oint_{\partial V}d\vb* A\cdot
                              \vb E(\vb* r, t)=
                              4\pi k Q_{\text{ges,V}}$\\
        Keine Magn. Monopole
                      & $\div \vb B(\vb* r, t)=0$
                            & $\ds\oint_{\partial V}d\vb* A\cdot 
                              \vb B(\vb* r, t) = 0 $\\
        Faraday'sche Gesetz 
                      & $\ds \curl \vb E(\vb* r , t)=-k''\partial_t
                        \vb B(\vb* r, t)$
                      & $\ds \oint_{\partial A}d\vb* l\cdot\vb 
                      E(\vb* r, t)=-k''\dv{t}\iint_Ad\vb* A\cdot \vb B(\vb* r, t)$\\
        Ampère'sche Gesetz  
          & $\ds \curl \vb B(\vb* r, t)=4\pi k'\vb* j(\vb* r, t)
            +\frac{k'}{k}\partial_t \vb E(\vb* r, t)$
          & $\ds\oint_{\partial A}d\vb* l\cdot \vb B(\vb* r, t)=
          \iint_{A}d\vb* A\cdot\qty(4\pi k'\vb* j(\vb* r, t)+
          \frac{k'}{k}\partial_t \vb E(\vb* r, t))$\\
        \end{tabular}
        }
        (SI: $k=\frac{1}{4\pi\epsilon_0}\quad k'=\frac{\mu_0}{4\pi}\quad
                  k''=1$)
        (Gauß: $k=1\quad k'=\frac{1}{c}\quad
             k''=\frac{1}{c}$) 
        \end{center}
        \begin{center}
          \sref{3.4.1$\sim$4}
        \end{center}
        
        \textbf{Notiz:} Die Integral-Form der Maxwell-Gleichungen folgt
        direkt aus beidseitig Integrieren der Differential-Form (über 
        Volumen für die Divergenz-Gesetze, über Oberfläche für die 
        Rotations-Gesetze) unter Anwendung der Gauß'sche und 
        Stoke'sche Gesetze.

  \item Welche Gleichung legt das Verhalten der aus Punktladungen
        zusammen gesetzten Quellen $\rho(\vb* r, t)$ und 
        $\vb* j(\vb* r, t)$ in den Feldern $\vb E(\vb* r, t)$
        und $\vb B(\vb* r, t)$ fest?
        \begin{center}
          Die allgemeine Lorentzkraft: 
          $\ds \vb* F_q=q\qty(\vb E+k''\vb* v\times\vb B)\smref{3.7.1}$
        \end{center}

  \item Man zeige mittels Kontinuitätsgleichung und Ampère-Gesetz, 
        dass der Maxwell'sche Verschiebungsstrom für Ladungserhaltung
        notwendig ist.
        $$\div \vb* j(\vb* r, t) = -\pdv{t}\rho(\vb* r ,t)$$
        $$\curl\vb B(\vb* r, t) \coloneqq 
        4\pi k'\vb* j(\vb* r ,t) + \vb* C(\vb* r, t)$$
        $$\text{Allgemein gilt:}\,\div\curl\vb* F(\vb* r, t)
        =\curl \div \vb* F(\vb* r, t)=0$$
        \begin{equation*}
          \begin{split}
          \ds \div\curl\vb B(\vb* r, t)
                      &\ds= \div \qty[
                      4\pi k'\vb* j(\vb* r, t)+\vb* C(\vb* r, t)]\\
                      &\ds= 4\pi k'\div\vb* j(\vb* r,t) +
                      \div \vb* C(\vb* r,t)\\
                      &\ds= 4\pi k'\qty(-\pdv{t}\rho(\vb* r, t)) 
                      +\div \vb* C(\vb* r,t)\stackrel{!}{=}0\\
          \Leftrightarrow\ds 4\pi k'\pdv{t}\rho(\vb* r, t) 
                      &\ds =\div \vb* 
                      C(\vb* r, t)\footnotemark
                      \stackrel{!}{=}\div \pdv{t}K\vb E(\vb* r, t)\\
                      &\ds =\ds \pdv{t} 
                      K4\pi k \rho(\vb* r, t)
          \Leftrightarrow K         
                      \stackrel{!}{=}\frac{k'}{k}\\
          \Leftrightarrow \vb* C(\vb* r,t)
                      &\coloneqq \frac{k'}{k}\pdv{t}\vb E(\vb* r, t)
          \end{split}
        \end{equation*}
        \footnotetext{Erinnere daß $\div \vb E(\vb* r, t)=4\pi k\rho(\vb* r ,t)$}      

  \item Wie lauten die Wellengleichungen für $\vb E(\vb* r, t)$ und
        $\vb B(\vb* r, t)$ im Vakuum?
        $$\qty(\Delta - \frac{1}{c^2}\pdv[2]{t})\vb E(\vb* r, t)=
        \qty(\Delta - \frac{1}{c^2}\pdv[2]{t})\vb B(\vb* r, t)=0
        \smref{3.5.2 und 3.5.3}$$

  \item Wie sehen Ebene-Wellen-Lösungen im Vakuum aus? Was weiß man
        über die Richtungen der Felder bezüglich der 
        Ausbreitungsrichtung?
        $$f(\vb* r , t)=f_r(\vb* r)f_t(t)$$
        $$\vb* k \,\bot\, \vb E \quad \vb* k\,\bot\,\vb B \quad
        \vb E\,\bot\, \vb B\smref{3.5.8$\sim$10}$$

  \item Wie erhällt man die Dispersionsrelation und wie lautet sie für
        elektromagnetische Wellen im Vakuum?
        \begin{center}
          Indem man die Lösung (gegebene Funktion) in 
          der Wellengleichung einsetzt und die
          triviale Terme gegen einander wegstreicht.
        \end{center}
        $$\dv{\omega(k)}{k}=c \quad\text{(im Vakuum)}\smref{3.5.7}$$

  \item Man gebe die Formeln für die Fourier-Hin- und Rücktransformation
        für eine Funktion $f(t)$ auf der $t$-Achse auf $\omega$-Achse und
        zurück an.
        $$\hat f(\omega, \alpha)=\int \frac{dt}{\sqrt{2\pi}}
              e^{-i\omega t}f(t, \alpha)\smref{3.6.12}$$
        $$f(t, \alpha)=\int \frac{d\omega}{\sqrt{2\pi}}
                         e^{+i\omega t}\hat f(\omega, \alpha)
                        \smref{3.6.11}$$
        \begin{center}
          ($\alpha$ stellt hier nicht zur Transformation relevante 
          sonstige Parameter dar, wie z.B. $\vb* r$)
        \end{center}
        \textbf{Notiz:} Hier wird explizit um die Fouriertransformation 
        nach Zeit gefragt. Dies darf nicht verwechseld werden mit die 
        Fouriertransformation von $\vb* x$-raum nach $\vb* k$-raum 
        wobei man über
        $d^3x$ bzw. $d^3k$ integriert wird \sref{3.6.17 und 3.6.18}.
        Welche Parameter transformiert werden hängt also von der Frage 
        ab!

  \item Man berechne die Fouriertransformation von $\delta(t-t_0)$.
        $$f(t)=\delta (t-t_0)$$ 
        $$F[f(t)]=\hat f(\omega)
        =\int\frac{dt}{\sqrt{2\pi}}e^{-i\omega t}\delta(t-t_0)
        =\frac{1}{\sqrt{2\pi}}e^{-i\omega t_0}
        \smref{3.9.7}$$ 

  \item Wie berechnet sich der Poynting-Vektor für reelle Felder? Was
        ist seine physikalische Bedeutung?
        $$\vb S(\vb* r,t)=\frac{1}{\mu_0}\vb E(\vb* r,t)\times\vb B(\vb* r,t)
        =\vb E(\vb* r,t)\times \vb H(\vb* r,t)
        \smref{3.7.8}$$
        $$\text{(Allgemein)}\quad\vb S(\vb* r,t)=
        \frac{1}{\mu_0}\Re[\vb E(\vb* r,t)]\times \Re[\vb B(\vb* r,t)]$$
        
        \begin{center}
          Die physikalische Bedeutung ist die von den 
          Feldern verursachte Energiefluß.
        \end{center}

  \clearpage
  \item Wie lauten die inhomogenen Maxwell-Gleichungen in Materie,
        charakterisiert durch Dielektrizitätskonstante $\epsilon_r$ und
        Permeabilität $\mu_r$?
        \begin{center}
          \begin{tabular}{rrcl}
            Inhomogen&$\div \vb D$ & $=$ &$\rho_f$\\
            &$\curl \vb H-\pdv{t}\vb D$ & $=$ &$\vb* j_f$\\
            Homogen&$\curl \vb E+\pdv{t}\vb B$ & $=$ &$0$\\
            &$\div \vb B$ & $=$ &$0$\\
          \end{tabular}
          \sref{4.1.5$\sim$8}
          $$\vb D(\vb* r,t)=\epsilon_r\epsilon_0\vb E(\vb* r,t) \quad
          \text{und}\quad\vb H(\vb* r, t)=\frac{1}{\mu_0\mu_r}
          \vb B(\vb* r,t)\smref{4.1.9}$$
        \end{center}

  \item Durch welche drei Sachverhalte unterscheiden sich 
        elektromagnetische Wellen in eienem Wellenleiter mit einfach
        zusammenhängenden rechteckigem Querschnitt von denen im Vakuum?
        \begin{enumerate}
          \item Die Dispersionsrelation ändert sich 
                ($\omega(\vb* k)\ne c\abs{\vb* k}$ i.A.)
          \item Es gilt i.A. \textbf{nicht} dass $\vb E\,\bot\,\vb B$
          \item Es gilt i.A. \textbf{nicht} dass $\vb E\,\bot\,\vb* k$
                oder dass $\vb B\,\bot\,\vb* k$
        \end{enumerate}
        \begin{center}
          \sref{§3.11}
        \end{center}

  \item Wie sind Phasen- und Gruppengeschwindigkeit definiert und was ist
        ihre physikalische Bedeutung?
        \begin{center}
          \begin{tabular}{ll}
            $v_{\text{ph}}=\frac{\omega(k)}{k}$ 
                      &Gescwhindigkeit mit 
                       den die Spitzen der Welle sich ausbreiten 
                       \\\\
            $v_{\text{gr}}=\dv{\omega(k)}{k}$
                      &Gewschwindigkeit mit
                       den die Knoten einer ebene Welle sich ausbreiten.\\
                      &Geschwindigkeit eines gesammten Wellenpackets.
                      \sref{3.12.9}\\
          \end{tabular}
        \end{center}
        \textbf{Notiz:} 
        $v_{\text{ph}}$ kann durchaus größer als $c$ sein. Dies ist aber
        kein Problem, denn die Phasengeschwindigkeit ``trägt'' keine
        Information. Für $v_{\text{gr}}$ gilt immer $v_{\text{gr}}\le c$!

  \item Man gebe zwei Formen für das Ohm'sche Gesetz an. Wie ist die
        Beziehung zwischen Leitfähigkeit $\sigma$ und Widerstand $R$ 
        für einen homogenen Leiter der länge $l$ und mit 
        Querschnitt $A$?
        \begin{enumerate}
          \item $U=IR\smref{3.2.1}$
          \item $\vb* j= \sigma \vb E\smref{3.2.2}$ 
            $\qty(\sigma=\frac{1}{\rho}\quad R=\frac{l}{\sigma A})$
        \end{enumerate}

  \item Wie entsteht eine komplexe, frequenzabhängige 
        Dielektrizitätskonstante $\epsilon(\omega)$ in einem Metal?
        $$\curl \vb H = \vb* j + \pdv{t}\vb D$$
        $$\curl \vb H -\qty(\sigma +\epsilon_0\epsilon_r\pdv{t})\vb E=0$$
        $$\curl \vb H -\qty(\sigma +\epsilon_0\epsilon_ri\omega)\vb E=0$$
        $$\curl \vb H +\epsilon_0 i\omega
        \qty(\epsilon_r-\frac{\sigma}{i\omega\epsilon_0})\vb E=0$$
        $$\curl \vb H +i\omega\epsilon_0 \epsilon(\omega)\vb E=0$$
        $$\epsilon(\omega)=\qty(\epsilon_r-\frac{\sigma}
        {i\omega\epsilon_0})\smref{4.2.5}$$

  \item Wie verhält sich die Dielektrizitätskonstante $\epsilon(\omega)$
        bei der Plasmafrequenz $\omega_P$? Was passiert physikalisch bei
        $\omega=\omega_P$ und bei $\omega>\omega_P$?
        $$
        \epsilon(\omega_P)=\epsilon_r\qty(1-\frac{\omega_P^2}{\omega^2})
        \smref{4.2.23}$$
        $$\Leftrightarrow\quad\epsilon(\omega_P)=0
        \quad\epsilon(\omega<\omega_P)<0 
        \quad\epsilon(\omega>\omega_P)>0$$
        $$\quad\omega_P=\sqrt{\frac{n_fe^2}
        {m\epsilon_0\epsilon_r}}\smref{4.2.31}$$
        \begin{tabular}{rl}
          $\ds\omega>\omega_p$ &$\Rightarrow$ 
          Metalle werden durchsichtig\\
          $\ds\omega\le\omega_p$& $\Rightarrow$ Exponentielle 
          Abfall von EM-Wellen im Medium
        \end{tabular}
        \begin{center}
          \sref{§4.2.3 S.123}
        \end{center}

\end{enumerate}

\newpage
\section{Grundkenntinsse Relativität und rel. Elektrodynamik}
\begin{center}
  \begin{tabular}{ll|ll|ll}
    \toprule
    Frage & Sicherheitsgrad & Frage & Sicherheitsgrad 
          & Frage & Sicherheitsgrad \\
    \midrule
    1& Sicher           & 6 & Sicher     & 11& Sicher\\ 
    2& Sicher           & 7 & Sicher     & 12& Sicher\\
    3& Sicher           & 8 & Sicher     & 13& Sicher\\
    4& Sicher           & 9 & Sicher     & 14&       \\
    5& Sicher           & 10& Sicher     & 15&       \\
    \bottomrule
  \end{tabular}
\end{center}

  \textbf{Wichtibe bemerkung:} Dieses Kapitel arbeitet schließlich
  mit Gauß'sche Einheiten!

\begin{enumerate}
  \item Was bedeutet das spezielle Relativitätsprinzip für physikalische
    Gesetze und die Lichtgeschwindigkeit?

    Die Lichtgeschwindigkeit ist eine Konstante in jedem Bezugssystem.
    Die physikalische Gesetze sollen in jedes Bezugssytem gleich sein,
    unterschiedliche Observierer können aber unterschiedliche Gründen
    angeben weshalb ein Erreignis statt findet. (z.B. kann ein Observierer
    im Ruhesystem einer Ladung nur Elektrische Kräfte sehen, während
    ein Observierer in ein Ruhesystem wo die Ladung sich bewegt auch
    magnetische Kräfte sieht.) \sref{§5.2.1}

  \item Wie lautet die Formel für den invarianten Raum-Zeit-Abstand ds in
    einem Koordinatensystem? Warum folgt aus der Invarianz von $ds$ die
    Invarianz der Lichgeschwindigkeit?
    
    Die invariante Raum-Zeit-Abstand ist gegeben durch:
    \begin{equation*}
      ds^2 = c^2 dt^2 - d\vb x^2
      \smref{5.2.5}
    \end{equation*}

      Für ein Teilchen, wie z.B. ein Photon, dass sich mit der 
      Lichtgeschwindigkeit bewegt soll gelten dass die Minkowski-Metrik
      verschwindet. Das heißt dass:
      \begin{equation*}
        (cdt)^2 - dx^2 = 0 
      \end{equation*}
      Wobei natürlich gilt dass $dx=vdt$ mit $v=c$.

      In System $S'$ gibt es jetzt die Parameter $t'$, $x'$ und $v'$
      mit $dx'=v'dt'$. Wir möchen zeigen, dass $v=v'$ falls $v=c$.
      Aus der Invarianz von $ds$ folgt dass $ds=ds'$, woraus folgt dass
      \begin{equation*}
        c^2dt^2-d\vb x^2 = ds^2 = 0 = ds'^2 = c^2dt'^2 - d\vb x'^2
      \end{equation*}
      Es muss also gelten, dass $c^2dt'^2 = dx'^2$ und deswegen folgt
      $c=v'\Rightarrow v=v'=c$, sodass sich die Geschwindigkeit 
      des Teilchens (bei Lichtgeschwindigkeit) als invariant ergibt.

  \item Welche Bedingungen stellt man an eigentliche orthochrone
    Lorentz-transformationen?

    \textbf{Definition:} Eine Lorentz-Transformation, bei der (Raum-) 
    Spiegelungen 
    ausgeschlossen sind und die Orientierung der Zeit erhalten ist, 
    wird als eigentliche, orthochrone Lorentz-Transformation bezeichnet.

    \begin{equation*}
      \Lambda^{0}_{\ 0} \ge 1 
      \quad \text{und}\quad
      \det \Lambda = 1 \smref(5.2.18)
    \end{equation*}
    \textbf{Notiz:} siehe auch 
    \url{https://de.wikipedia.org/wiki/Lorentz-Transformation} (suche 
    nach orthochron mit deinem Browser)

  \item Wie lautet die Lorentz-Transformation für einen Boost in 
    $z$-Richtung? Wie sieht die Inverse aus?
    
    Sei $S$ ein Ruhesystem und bewege $S'$ sich mit relativen 
    Geschwindigkeit $v$ in $z$-Richtung. Man definiere die (einheitslosen)
    Größen:
    $\beta = \frac{v}{c}$, $\gamma=\qty(1-\frac{v^2}{c^2})^{-\frac{1}{2}}$
    \sref{5.2.45}
    
    Für die Hintransformation gilt:
    \begin{equation*}
      x'^\mu = \Lambda^\mu_{\ \nu} x^\nu 
      \text{ mit } 
      \Lambda^\mu_{\ \nu}=
      \begin{pmatrix}
        \gamma       & 0 & 0 & -\beta\gamma\\
        0            & 1 & 0 & 0           \\
        0            & 0 & 1 & 0           \\
        -\beta\gamma & 0 & 0 & \gamma      \\
      \end{pmatrix}
      \smref{5.2.46}
    \end{equation*}
    Für die Rücktransformation gilt:
    \begin{equation*}
      x^\mu = \qty(\Lambda^{-1})^\mu_{\ \nu} x'^\nu 
      \text{ mit } 
      \Lambda^\mu_{\ \nu}=
      \begin{pmatrix}
        \gamma       & 0 & 0 & \beta\gamma\\
        0            & 1 & 0 & 0           \\
        0            & 0 & 1 & 0           \\
        \beta\gamma  & 0 & 0 & \gamma      \\
      \end{pmatrix}
      \smref{5.2.48}
    \end{equation*}

    Man erinnere sich daran dass:
      $(\Lambda^{-1})^\mu_{\ \nu}
      =g_{\mu\sigma}\Lambda^{\sigma}_{\ \gamma}g^{\gamma\nu}
      =\Lambda_{\nu}^{\ \mu}
      \smref{5.2.25}$

  \item Man schreibe den kontravarianten Vierer-Vektor für ein
    Raum-Zeit-Ereignis x. Wie transformiert sich ein kontravarianter
    Lorentz-Vierer-Vektor?

    \begin{equation*}
      x^\mu =
      (ct, x, y, z)^T
      \smref{5.2.3}
    \end{equation*}
    \begin{equation*}
    \qquad x'^\mu = \Lambda^\mu_{\ \nu} x^\nu 
    \qquad x^\mu = (\Lambda^{-1})^\mu_{\ \nu} x'^\nu 
    \smref{5.2.12}
    \end{equation*}

  \item Wie erhält man den kovarianten Vierer-Vektor für x mit dem
    metrischen Tensor?

    \begin{equation*}
      x_\mu = g_{\mu\nu}x^\nu=
      (ct, -x, -y, -z)^T
      \smref{5.2.8}
    \end{equation*}

  \item Wie sieht das invariante Skalarprodukt zwischen zwei 
    Vierer-Vektoren in einem Koordinaten-System aus, ohne und mit
    dem metrischen Tensor geschrieben?
    
    Seien $v^\mu$, $w^\mu$ Vierervektoren der Form: 
    $(x_0, x_1, x_2, x_3)^T$ mit $x=v,w$

    \begin{equation*}
      \begin{split}
      \langle v^\mu , w^\mu \rangle 
      &\coloneqq v^\mu w_\mu = 
      v^\mu g_{\mu\nu} w^\nu
      =
      \begin{pmatrix}
        v_0 \\ v_1 \\ v_2 \\ v_3
      \end{pmatrix}
      \begin{pmatrix}
        1 & 0  & 0  &  0\\
        0 & -1 & 0  &  0\\
        0 & 0  & -1 &  0\\
        0 & 0  & 0  & -1\\
      \end{pmatrix}
      \begin{pmatrix}
        w_0 \\ w_1 \\ w_2 \\ w_3
      \end{pmatrix}\\
      &=v_0w_0 - (v_1w_1 + v_2w_2 + v_3w_3)
      \smref{5.2.6}
      \end{split}
    \end{equation*}

    Man bemerke dass $w_\mu=(w_0, -w_1, -w_2, -w_3)^T$

    Für $x_0=ct_x$ und $x_{i\in\{1,2,3\}}$ (also im Koordinatensystem) 
    folgt natürlich:

    \begin{equation*}
      v^\mu w_\mu = c^2(t_vt_w) - (v_xw_x + v_yw_y + v_zw_z)
    \end{equation*}

  \item Warum ist die Gescwhindigkeit difineiert als $v^\mu=\dv t x^\mu$
    kein Lorentz-Vektor? Welche verwandtete Geschwindigkeit ist ein
    Lorentz-Vektor?

    $v^\mu$ ist kein Vierer-Vektor weil es nicht sich nicht
    wie ein Lorentz-Vektor. Das heißt, dass:
    \begin{equation*}
      \dv {t'} x'^\mu \neq \Lambda^{\mu}_{\ \nu}\dv t x^\mu
      \smref{5.2.61}
    \end{equation*}

    $u^\mu$, die \textbf{Weltgeschwindigkeit}, 
    ist schon ein Lorentz-Vektor,
    mit $u^\mu=\dv \tau x^\mu$ und
    $$d\tau=\frac{1}{c}ds=\gamma^{-1}dt \smref{5.2.54\,\&\,55}$$

  \clearpage
  \item Wie lauten die Komponenten des Vierer-Potenitals und des
    Vierer-Stroms in einem gegebenen Koordinatensystem? Wie schreibt
    sich die Wellengleichung damit?

    \begin{equation*}
      A^\mu(x^\alpha)=
      \begin{pmatrix}
        \Phi\\ A_x\\ A_y\\ A_z
      \end{pmatrix}
      =
      \begin{pmatrix}
        \Phi(x^\alpha) \\ \vb A(x^\alpha)
      \end{pmatrix}
      \smref{5.3.8}
    \end{equation*}
    \begin{equation*}
      j^\mu(x^\alpha) =
      \begin{pmatrix}
        c\rho\\ j_x\\ j_y\\ j_z
      \end{pmatrix}
      =
      \begin{pmatrix}
        c\rho(x^\alpha) \\ \vb j(x^\alpha)
      \end{pmatrix}
      \smref{5.3.1}
    \end{equation*}

    Alternativ gilt auch $j^\mu=\rho \dv{x^\mu}{t}
    =\frac{\rho}{\gamma}u^\mu \smref{5.3.2\ \&\,4}$

    Sei 
    $\square
    = -\qty(\frac{1}{c^2}\partial_t^2-\Delta)
    = -\qty(\pdv[2]{x_0}-\pdv[2]{x_1}-\pdv[2]{x_2}-\pdv[2]{x_3})
    = -\partial_\mu\partial^\mu \smref{5.3.11}$
    
    Dann gilt für die Wellengleichung von $A^\mu$:
    \begin{equation*}
      \square A^\mu = -\frac{4\pi}{c}j^\mu 
      \quad\text{bzw.}\quad
      \partial_\nu\partial^\nu A^\mu = \frac{4\pi}{c}j^\mu 
      \smref{5.3.12\,\&\,13}
    \end{equation*}

    \textbf{Notiz:} $\pdv{x^\mu}=\partial^\mu=(\frac{1}{c}\dv t, \nabla)$

  \item Man gebe den Feldstärke-Tensor als Funktion des Vierer-Potentials
    und Matrix mit den Komponenten von $\vb E$ und $\vb B$ an.

    \begin{equation*}
      F^{\mu\nu}=\partial^\mu A^\nu - \partial^\nu A^\mu=
      \begin{pmatrix}
        0   & -E_x & -E_y & -E_z\\ 
        E_x & 0    & -B_z & B_y \\ 
        E_y & B_z  & 0    & -B_x\\ 
        E_z & -B_y & B_x  & 0   \\ 
      \end{pmatrix}
    \smref{5.3.16}
    \end{equation*}

  \item Wie lauten die homogenen Maxwell-Gleichungen in forminvarianter
    Schreibweise?

    \begin{equation*}
      \partial^\mu F^{\nu\lambda} + 
      \partial^\lambda F^{\mu\nu} + 
      \partial^\nu F^{\lambda\mu} + 
      =0 \smref{5.3.23}
    \end{equation*}
    Oder auch:
    \begin{equation*}
      \partial_\mu \tilde F^{\mu\nu}
      =0 \smref{5.3.28}
    \end{equation*}

    Dabei ist $\tilde F^{\mu\nu}=\frac{1}{2}
    \epsilon^{\mu\nu\eta\gamma}F_{\eta\gamma}$
    
    Mit \sref{5.3.25}
    \begin{equation*}
      \epsilon^{\mu\nu\eta\gamma}=
      \begin{cases}
        1&\text{gerade Permutation bzw. zykische Vertauschung}\\
        -1&\text{ungerade Permutation bzw. 
        antizyklische Vertauschung}\\
          0&\text{nicht alle verschieden}
      \end{cases}
    \end{equation*}

    Es folgt daraus direkt dass
    \begin{equation*}
      \begin{array}{lll}
        \div B = 0 & (\nu = 0) & \smref{5.3.27} \\
        \curl E = -\frac{1}{c}\partial_t B & (\nu = 1,2,3) 
                                           & \smref{5.3.27} 
      \end{array}
    \end{equation*}

  \item Wie lauten die inhomogenen Maxwell-Gleichungen in forminvarianter
    Schreibweise?

    Wegen $\partial_\mu A^\mu=0 \smref{5.3.15}$ 
    (folgt aus Kontinuitätsgleichung), gilt auch
    
    \begin{equation*}
      \partial_\mu F^{\mu\nu}
      =\partial_\mu\qty[\partial^\mu A^\nu-\partial^\nu A^\mu]
      =\partial_\mu\partial^\mu A^\nu - \partial^\nu 
      \underbrace{\partial_\mu A^\mu}_{0}
      =-\square A^\nu = \frac{4\pi}{c}j^\nu \smref{5.3.20} 
    \end{equation*}
    Es folgt daraus direkt dass
    \begin{equation*}
      \begin{array}{lll}
        \div \vb E = 4\pi \rho &(\nu=0) &\smref{5.3.21}\\
        \curl \vb B = \frac{4\pi}{c}\vb j + \frac{1}{c}\partial_t \vb E
                    &(\nu=1,2,3) &\smref{5.3.22}
      \end{array}
    \end{equation*}

  \item Wie schreibt sich die Lorentz-Kraft für eine Ladung q auf
    forminvariante Weise?

    \begin{equation*}
      K^\mu=\dv \tau p^\mu =m \dv \tau u^\mu = \frac{q}{c} F^{\mu\nu}u_\nu
      \smref{5.4.53}
    \end{equation*}

  \item Man gebe das kirchhoff'sche Beugungsintegral [für
    Lichteinfall] auf eine Blendenöffnung bei weit 
    entfernter Lichtquelle und Senkrechtem Einfall an.

  \item Was ergibt sich in der Fraunhofer-Näherung? Wann gilt diese?

\end{enumerate}

\end{document}
