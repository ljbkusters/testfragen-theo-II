\documentclass{scrartcl}
\usepackage{enumitem}
\usepackage{parskip}
\usepackage[margin=2.5cm]{geometry} 
\usepackage[utf8]{inputenc}
\usepackage[ngerman]{babel}
\usepackage{physics}
\usepackage{hyperref}
\usepackage{booktabs}
\usepackage{bbold}
\usepackage{bm}
\usepackage{graphicx}
\usepackage{mathtools}
\usepackage{amssymb}
\usepackage{amsfonts}
\usepackage{cleveref}

\renewcommand{\thefootnote}{\fnsymbol{footnote}}

\newcommand{\rr}[1]{\frac{#1}{\abs{\vb*{r}-\vb*{r}'}}} 
\newcommand{\ds}{\displaystyle} 

% reference skript pages
\newcommand{\sref}[1]{(Skript #1)} 
\newcommand{\smref}[1]{\quad\text{(Skript #1)}} 


\title{Probeklausur zur Theo II}
\begin{document}
\textbf{Aufgabe 1}\hfill[60 Punkte (15 $\times$ 2)]

Siehe die Fragensammlung

\textbf{Aufgabe 2}\hfill[8+7=15 Punkte]

Betracten Sie eine Kugel mit Radius R, deren Mittelpunkt im Ursprung des
Koordinatensystems liegt. Die Kugeloberfläche trägt die Flächenladungsdichte
$\rho_F(\theta,\phi)=\rho_0 \cos\theta$. Im Innenraum ($\abs{\vec x}<R$)
und im Außenraum ($\abs{\vec{x}}>R$) der Kugel befindet sich Vakuum.
\begin{enumerate}[label=\alph*)]
  \item Bestimmen Sie eine geignete Näherung für das Potential 
    $\Phi(\vec x)$ in $\abs{\vec x}\gg R$ und daraus
    das $\vec E$-Feld

  \item Bestimmen Sie eine geignete Näherung für das Potential 
    $\Phi(\vec x)$ in der Nähe des Ursprungs $\abs{\vec x}\ll R$ und daraus
    das $\vec E$-Feld
\end{enumerate}
Es sollen jeweils nur die führenden, d.h.\ die ersten nicht verschwindenden
Terme der entsprechenden Entwicklung berechnet werden.

\textbf{Aufgabe 3}\hfill[3+8+4=15 Punkte]
Betrachten Sie eine zeitlich veränderliche Stromdichte $J(\vec x,t)$, deren
räumliche Ausdehnung so klein ist, dass Sie in komplexer Schreibweise durch
$\vec J(\vec x,t)=J_0 e^{-i\omega t}\delta(\vec x)\vec e_z$ approximiert
werden kann. ($J_0$ und $\omega$ konstant). Die Quelle befindet sich 
im Vakuum.
\begin{enumerate}[label=\alph*)]
  \item Zeigen Sei ausgehend von einem geeigneten integralausdruck, dass
    das zugehörige Vektropotential durch
    \begin{equation*}
      \vec A(\vec x, t)
      =J_0 e^{-i\omega t}\frac{\mu_0}{4\pi}\frac{e^{-ikr}}{r}\vec e_z
    \end{equation*}
    mit $r=\abs{\vec x}$ gegeben ist.

  \item Berechnen Sie mit dem Ergebnis aus a) das $\vec B$-Feld und
    daraus da $\vec E$-Feld.

  \item Hier sollen nur noch die Teile der in b) berechneten Felder
    betrachtet werden, die nicht stärker als $1/r$ abfallen, d.h. man
    betrachtet eine Näherung für die Felder weit weg von der Quelle.
    Berechnen Sie damit den zeitlich gemittelten Poyntingvektor und
    die zeitlich gemittelte Leistung $\dv {P}{\theta}$, die unter dem
    Winkel $\theta$ zur $z$-Achse abgestrahlt wird.
\end{enumerate}

\end{document}
