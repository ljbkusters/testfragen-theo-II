\documentclass{scrartcl}
\usepackage{enumitem}
\usepackage{parskip}
\usepackage[margin=2.5cm]{geometry} 
\usepackage[utf8]{inputenc}
\usepackage[ngerman]{babel}
\usepackage{physics}
\usepackage{hyperref}
\usepackage{booktabs}
\usepackage{bbold}
\usepackage{bm}
\usepackage{graphicx}
\usepackage{mathtools}
\usepackage{amssymb}
\usepackage{amsfonts}
\usepackage{cleveref}

\renewcommand{\thefootnote}{\fnsymbol{footnote}}

\newcommand{\rr}[1]{\frac{#1}{\abs{\vb*{r}-\vb*{r}'}}} 
\newcommand{\ds}{\displaystyle} 

% reference skript pages
\newcommand{\sref}[1]{(Skript #1)} 
\newcommand{\smref}[1]{\quad\text{(Skript #1)}} 


\title{Probeklausur zur Theo II}
\begin{document}
\textbf{Aufgabe 1}\hfill[60 Punkte (15 $\times$ 2)]

Siehe die Fragensammlung

\textbf{Aufgabe 2}\hfill[8+7=15 Punkte]

Betracten Sie eine Kugel mit Radius R, deren Mittelpunkt im Ursprung des
Koordinatensystems liegt. Die Kugeloberfläche trägt die Flächenladungsdichte
$\rho_F(\theta,\phi)=\rho_0 \cos\theta$. Im Innenraum ($\abs{\vec x}<R$)
und im Außenraum ($\abs{\vec{x}}>R$) der Kugel befindet sich Vakuum.

Es sollen jeweils nur die führenden, d.h.\ die ersten nicht verschwindenden
Terme der entsprechenden Entwicklung berechnet werden.
\begin{enumerate}[label=\alph*)]
  \item Bestimmen Sie eine geignete Näherung für das Potential 
    $\Phi(\vec x)$ in $\abs{\vec x}\gg R$ und daraus
    das $\vec E$-Feld

    \textbf{Lösung} 

    Man verallgemeinere die Ladungsdichte zunächst mit die 
    $\delta$-Distribution.
    \begin{equation}
      \rho(r,\theta) = \rho_0 \cos\theta \delta(r-R)
    \end{equation}

    \textbf{Potential: Karthesische Entwicklung} 

    Man sehe direkt aus symmetrie Grunden dass $Q_{ges}=0$ 
    (wegen $\rho\propto\cos\theta$).
    Das heißt, dass der führenden Term höchstens das Dipolmoment sein kann.
    Für das Dipolmomnet gilt:
    \begin{equation}
      p_i = \int d^3x x_i \rho(\vec x)
    \end{equation}
    Dabei ist $x_1=x=r\sin\theta\cos\phi$, $x_2=y=r\sin\theta\sin\phi$,
    $x_3=z=r\cos\theta$

    Aus symmetrie Grunden würde es Sinn machen wenn $p_x$ und $p_y$
    verschwinden. Dies ist leicht zu überprüfen denn

    \begin{equation*}
      p_x=\int d^3r r\sin\theta\cos\phi \rho (\vec r)
      \propto \int_{0}^{2\pi}\cos\phi = 0
    \end{equation*}
    und analog
    \begin{equation*}
      p_y=\int d^3r r\sin\theta\sin\phi \rho(\vec r) 
      \propto \int_{0}^{2\pi}\sin\phi = 0
    \end{equation*}

    Es würde auch Sinn machen, dass $p_z$ nicht verschwindet, wegen der
    Polarität der Ladungsdichte.
    \begin{equation*}
      \begin{split}
        p_z
        &=\int d^3 r \underbrace{(r \cos\theta)}_{z} 
        \underbrace{\rho_0 \cos\theta \delta(r-R)}_{\rho(\vec r)}\\
        &=
        \rho_0
        \int_0^{2\pi}d\phi
        \int_{-1}^{1}d\cos\theta 
        \int_0^{\infty} dr r^3 \cos^2\theta \delta(r-R)\\
        &=
        2\pi\rho_0
        \int_{-1}^{1}d\cos\theta \cos^2\theta 
        \int_0^{\infty} dr r^3 \delta(r-R)\\
        &=
        2\pi\rho_0
        \qty[\frac{2}{3}]
        \qty[R^3]
        =
        \frac{4\pi}{3}R^3 \rho_0
        =V_k \rho_0
      \end{split}
    \end{equation*}
    Sodass $\vec p=V_k\rho_0 \vu e_z$ mit $V_k$ das Volumen der Kugel
    (bemerke, dass, obwohl nicht zufällig, das Kugelvolumen nur ein
    mathematisches Ergibnis ist, und keine tiefere Bedeutung hat).

    Das Potential wird nun bis zur führende Ordnung gegeben durch
    \begin{equation}
      \Phi(\vec r)= k\frac{\vec r\cdot\vec p}{\abs{\vec r}^3}
      = kV_k\rho_0 \frac{r\cos\theta}{r^3} 
      = \frac{1}{4\pi\epsilon_0} V_k\rho_0 \frac{\cos\theta}{r^2} 
    \end{equation}

    \textbf{Potential: Spährische Entwicklung} 
    
    Man Schreibe $\rho(\vec r)$ zunächst als linearkombination von
    Kugelflächenfunktionen.

    \begin{equation*}
      \rho(\vec r)
      =\rho_0\cos\theta\delta(r-R)
      =\rho_0\sqrt{\frac{4\pi}{3}}Y_{10}(\theta,\phi)\delta(r-R)
    \end{equation*}
    Wobei wir benutzt haben dass $Y_{10}=\sqrt{\frac{3}{4\pi}}\cos\theta$

    Das Potential wird für $r\gg0$ gegeben durch
    \begin{equation}
      \Phi(\vec r)
      =k\sum_{lm}\frac{4\pi}{2l+1}\frac{q_{lm}}{r^{l+1}}Y_{lm}(\theta,\phi)
      \quad l\in\mathbb{N}_0\quad m\in\mathbb{Z}\quad -l\le m\le l
    \end{equation}
    mit
    \begin{equation}
      q_{lm}=\int d^3 r r^l \rho(\vec r) Y^*_{lm}(\theta,\phi)
    \end{equation}
    
    Wir möchten jetzt die Orthogonalität der Kugelflächenfunktionen
    anwenden. Es gilt
    \begin{equation}\label{eq:orthogonal}
      \int_\Omega d\Omega Y_{lm}(\Omega) Y^*_{l'm'}(\Omega)
      =\delta_{ll'}\delta_{mm'}
    \end{equation}
    Wobei man über den gesammten Raumwinkel $\Omega$ integriert.
    
    Wegen der Orthogonalität aus Formel~\ref{eq:orthogonal} zusammen mit
    unser Ladungsverteilung sieht man sofort, dass alle $q_{lm}=0$ mit 
    $l\neq 1$, $m\neq0$.

    \begin{equation*}
      \begin{split}
        q_{10}=\int dr r^2 \int d\Omega (r^1) \rho_0 \sqrt{\frac{4\pi}{3}}
        Y_{10}Y^*_{10}\delta(r-R)=\rho_0 \sqrt{\frac{4\pi}{3}}R^3
      \end{split}
    \end{equation*}

    Sodass sich für das Potential ergibt, dass
    \begin{equation}
      \begin{split}
      \Phi(r,\theta) 
      &= 
      k
      \qty(\frac{4\pi}{3})
      \qty(\rho_0 \sqrt{\frac{4\pi}{3}}R^3)
      \frac{1}{r^2}
      Y_{10}(\theta,\phi)\\
      &= 
      k
      \underbrace{\frac{4\pi}{3}R^3}_{V_k}\rho_0
      \frac{1}{r^2}
      \sqrt{\frac{4\pi}{3}}\sqrt{\frac{3}{4\pi}}\cos\theta\\
      &=\frac{1}{4\pi\epsilon_0}V_k\rho_0 \frac{\cos\theta}{r^2} 
      \end{split}
    \end{equation}

    \textbf{E-Feld} 

    Nun das wir wissen, dass 
    $\Phi(r,\theta)
    =\frac{V_k\rho_0}{4\pi\epsilon_0}\frac{cos\theta}{r^2}$ können wir das
    elektrische Feld finden.
    \begin{equation}
      \vec E(\vec r)=-\grad\Phi(\vec r)
    \end{equation}
    Das Problem lässt sich am einfachsten lösen indem wir $\nabla$ in
    Kugelkoordinaten anwenden. Es gilt
    \begin{equation*}
    \nabla=\nabla_k
    =\vu e_r \partial_r 
    + \vu e_\theta \frac{1}{r}\partial_\theta
    + \vu e_\phi \frac{1}{r\sin\theta}\partial_\phi
    \end{equation*}

    sodass sich für das Elektrische Feld ergibt dass
    \begin{equation}
      \vec E (r,\theta)= 
      -\frac{V_k\rho_0}{4\pi\epsilon_0}
      \qty[-2 \frac{\cos\theta}{r^3}\vu e_r 
           -\frac{\sin\theta}{r^3}\vu e_\theta]
           = \frac{V_k\rho_0}{4\pi\epsilon_0}\frac{1}{r^3}
           (2\cos\theta \vu e_r + \sin\theta \vu e_\theta)
    \end{equation}

  \item Bestimmen Sie eine geignete Näherung für das Potential 
    $\Phi(\vec x)$ in der Nähe des Ursprungs $\abs{\vec x}\ll R$ und daraus
    das $\vec E$-Feld

    \textbf{Lösung} 
    
    \textbf{Potential: Sphärische Entwicklung} 

    Mit der karthesische Entwicklung findet man hier keine Lösung, denn
    sie Funktioniert nur bei $r\gg0$. Man
    muss also eine spährische Multipol-Entwicklung durchführen.

    Für die spährische Multipol-Entwicklung nähe $r=0$ gilt nun
    \begin{equation}
      \Phi(\vec r)=k\sum_{lm}\frac{4\pi}{2l+1}r^lq_{lm}Y_{lm}(\theta,\phi)
    \end{equation}
    mit
    \begin{equation}
      q_{lm}=\int d^3 r \frac{1}{r^{l+1}}\rho(\vec r)Y^*_{lm}(\theta,\phi) 
    \end{equation}

    Sowie in Aufgabe a) verschwinden alle $q_{lm}$ außer $q_{10}$.
    \begin{equation*}
      \begin{split}
        q_{10}
        &=
        \int d^3r \frac{1}{r^{1+1}}\delta(r-R)
        \sqrt{\frac{4\pi}{3}}Y_{10}Y^*_{10} \\
        &= \rho_0 \sqrt{\frac{4\pi}{3}}
        \underbrace{\int dr \frac{r^2}{r^2} \delta(r-R)}_{1}
        \underbrace{\int d\Omega Y_{10}(\Omega)Y^*_{lm}(\Omega)}_{1}
      \end{split}
    \end{equation*}
    Daraus ergibt sich dann das Potential
    \begin{equation}
      \begin{split}
        \Phi(\vec r)
        &=k \frac{4\pi}{3} r \sqrt{4\pi/3} \sqrt{3/4\pi} \cos\theta\\
        &=\frac{4\pi}{4\pi\epsilon_0}\frac{\rho_0}{3}r\cos\theta
        =\frac{\rho_0}{3\epsilon_0}r\cos\theta 
      \end{split}
    \end{equation}

    \textbf{E-Feld}
    
    Das $\vec E$-Feld lässt sich einfach finden wie in a)
    \begin{equation}
      \vec E(\theta) = -\frac{\rho_0}{3\epsilon_0}
      \qty(\cos\theta \vu e_r - \sin\theta \vu e_\theta) 
      =-\frac{\rho_0}{3\epsilon_0}\vu e_z
    \end{equation}
    (Alternativ hätte man auch sehen können dass 
    $\Phi\propto r\cos\theta=z$ und dann die karthesische definition von
    $\nabla$ einsetzen können)

    \vfill
    \textbf{Die Korrektheit der Lösungen 
      wurde überprüft und ist garantiert.}

    (Nach Aufgabe $2.3.13$ aus
    ``Grundkurs Theretische Physik 3: Elektrodynamik'' von Wolfgang Nolting
    [10. Auflage])

\end{enumerate}

\newpage
\setcounter{equation}{0}
\textbf{Aufgabe 3}\hfill[3+8+4=15 Punkte]
Betrachten Sie eine zeitlich veränderliche Stromdichte $J(\vec x,t)$, deren
räumliche Ausdehnung so klein ist, dass Sie in komplexer Schreibweise durch
$\vec J(\vec x,t)=J_0 e^{-i\omega t}\delta(\vec x)\vec e_z$ approximiert
werden kann. ($J_0$ und $\omega$ konstant). Die Quelle befindet sich 
im Vakuum.
\begin{enumerate}[label=\alph*)]
  \item Zeigen Sei ausgehend von einem geeigneten Integralausdruck, dass
    das zugehörige Vektropotential durch
    \begin{equation*}
      \vec A(\vec x, t)
      =J_0 e^{-i\omega t}\frac{\mu_0}{4\pi}\frac{e^{-ikr}}{r}\vec e_z
    \end{equation*}
    mit $r=\abs{\vec x}$ gegeben ist.

  \item Berechnen Sie mit dem Ergebnis aus a) das $\vec B$-Feld und
    daraus da $\vec E$-Feld.

  \item Hier sollen nur noch die Teile der in b) berechneten Felder
    betrachtet werden, die nicht stärker als $1/r$ abfallen, d.h. man
    betrachtet eine Näherung für die Felder weit weg von der Quelle.
    Berechnen Sie damit den zeitlich gemittelten Poyntingvektor und
    die zeitlich gemittelte Leistung $\dv {P}{\theta}$, die unter dem
    Winkel $\theta$ zur $z$-Achse abgestrahlt wird.

  \textbf{Aufgabe kommt aus Nachklausur WS18/19, dazu wurde eine
  Lösung veröffentlicht, womit diese Lösung überprüft wurde}
\end{enumerate}

\end{document}
