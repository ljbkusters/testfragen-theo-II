\documentclass{scrartcl}
\usepackage{enumitem}
\usepackage{parskip}
\usepackage[margin=2.5cm]{geometry} 
\usepackage[utf8]{inputenc}
\usepackage[ngerman]{babel}
\usepackage{physics}
\usepackage{hyperref}
\usepackage{booktabs}
\usepackage{bbold}
\usepackage{bm}
\usepackage{graphicx}
\usepackage{mathtools}
\usepackage{amssymb}
\usepackage{amsfonts}
\usepackage{cleveref}

\renewcommand{\thefootnote}{\fnsymbol{footnote}}

\newcommand{\rr}[1]{\frac{#1}{\abs{\vb*{r}-\vb*{r}'}}} 
\newcommand{\ds}{\displaystyle} 

% reference skript pages
\newcommand{\sref}[1]{(Skript #1)} 
\newcommand{\smref}[1]{\quad\text{(Skript #1)}} 


\title{Herleitungen für Fragenblatt zur Theo II}
\begin{document}
  \author{Luc Kusters}
\date{WS 19/20}
\maketitle

\noindent
Die Lösungen die hier zusammengefasst wurden, kommen von mir und
aus die Theo II Whatsappgruppe. 
\textbf{Weder absolute Vollständichkeit noch absolute Korrektheit wird garrantiert!} Aber wo ich mir unsicher war habe ich Sachen nachgefragt, oder 
mittels dem Skript geklärt.
Bei Anmerkungen, Fragen, Korrekturen oder sonstiges, könnt ihr mich gerne 
per Whatsapp oder per email erreichen:\\ 
ljbkusters@gmail.com \\

\noindent
Vielen Dank an jedem, der mitgeholfen hat! Insbesondere:
\begin{itemize}
  \item Prof. Honerkamp / Dr. Mück für die Fragen
  \item ``Der Andere'' für Seine Bilder mit Lösungen in der Whatsappgruppe
  \item Lais für seine Aufmerksamkeit bei Fragen 2.7 und 2.8, wo immer noch Konstanten falsch definiert waren.
  \item ``JP'' für Seine Aufmerkungen zur verschiedene Fragen die die 
    Formulierung von verschiedene Antworten verbessert haben.
\end{itemize}

 
  \newpage

  \textbf{Frage 1.1}
    \begin{equation*}
      \begin{split}
        \grad \rr{1} 
          &= \partial_i \vu e_i \frac{1}{\abs{r_i \vu e_i - r_i' \vu e_i}}
          \quad\text{Notation: } \vb* r = r_i \vu e_i = \sum_i r_i \vu e_i\\
          &= \vu e_i \partial_i \frac{1}{\sqrt{(r_i-r_i')_i^2}}
          \quad\text{Notation: }(r_i-r'_i)^2_i=\sum_i(r_i-r'_i)^2\\
          &= 
          -\frac{2(r_i-r_i')_i}{2\qty[\sqrt{(r_i-r_i')_i^2}]^{3}}\vu e_i \\
          &= -\frac{\vb* r - \vb* r'}{\abs{\vb* r - \vb* r'}^3} 
      \end{split}
    \end{equation*}

  \textbf{Frage 1.2}
    für $\vb* r \neq \vb* r'$
    \begin{equation*}
      \begin{split}
        \Delta \rr{1}
        &=-\div\frac{\vb* r - \vb* r'}{\abs{\vb* r - \vb* r'}^3}\\  
        &=-\partial_i \vu e_i \cdot 
        \frac{r_i \vu e_i - r'_i \vu e_i}
        {\qty[(r_i - r'_i)_i^2]^{3/2}}\\  
        &=-\qty(
        \frac{3}{\qty[(r_i - r'_i)_i^2]^{3/2}}
        +(r_i-r'_i)_i\qty(-\frac{3}{2})\frac{2(r_i-r'_i)}
        {{\qty[(r_i-r'_i)_i^2]}^{5/2}}
        )\\  
        &=-\qty(
        \frac{3}{\qty[(r_i-r'_i)_i^2]^{3/2}} 
        + \frac{-3(r_i-r'_i)_i^2}{\qty[(r_i-r'_i)_i^2]^{5/2}}
        )\\
        &=0
      \end{split}
    \end{equation*}
    für $\vb* r=\vb* r'$
    \begin{equation*}
      \rr{1}\to \infty\quad \text{und}\quad \Delta\rr{1}\ \text{undefiniert}
    \end{equation*}
    Dies ist sind genau die Eigenschaften der $\delta$-Distribution.

    Man weißt also dass $\Delta \rr{1}=-C\delta(\vb* r - \vb* r')$
    Man möchte C noch finden:

    \begin{equation*}
      \int_V dV \Delta \rr{1} 
      = -\int_V dV \div \frac{\vb* r - \vb* r'}{\abs{\vb* r - \vb* r'}^3} 
      = -\int_{\partial V} 
      d\vb* A \frac{\vb* r - \vb* r'}{\abs{\vb* r - \vb* r'}^3}
    \end{equation*}
    Führe die Koordinaten $\tilde {\vb* {r}} = \vb* r-\vb* r'$ ein.
    $d\tilde {\vb* A} =dA \vu { e}_{\tilde r}$. Bemerke dass 
    $(\vb* r-\vb*r')\cdot d\vb* A= \tilde{\vb* r}\cdot d\tilde{\vb* A}$
    \begin{equation*}
      = -\int_{\partial V} 
      d\tilde{\vb*{A}} \frac{\tilde{\vb* {r}}}{\abs{\tilde{\vb*{r}}}^3}
      = -\int_{\partial V} 
      d\tilde \Omega \tilde r^2 \frac{\tilde r}{\tilde r^3} = -4\pi
    \end{equation*}
    Es folgt dann aus der Definition der $\delta$-Distribution dass
    \begin{equation*}
      -4\pi = \int dV \Delta \rr{1} = -\int dV C\delta(\vb* r - \vb* r')
      \Rightarrow C=4\pi
    \end{equation*}
    Sodass insgesammt gilt:
    \begin{equation*}
      \Delta \rr{1} = -4\pi \delta (\vb* r - \vb* r')
    \end{equation*}

  \newpage
  \textbf{Frage 2.1} 
   
    \begin{equation*}
      \text{Zz:}\quad
      \rr{1}= \frac{1}{\abs{\vb*r}}
      +\frac{\vb*r\cdot\vb*r'}{\abs{\vb*r}^3} 
      +\frac{1}{2}\sum_{i,j=1}^3 \frac{3r_ir_j-\delta_{ij}\vb*r^2}{r^5}
      r_i'r_j'
    \end{equation*}

    Eine allgemeine mehrdimensionale Taylorentwicklung wird gegeben durch:

    \begin{equation*}
      f(\vb*\alpha, \vb*\beta)
      =\qty(\exp(\vb*\beta\cdot\nabla_{\vb*\beta'})
      f(\vb*\alpha, \vb*\beta'))
      \bigg|_{\vb*\beta=\vb*\beta_0}
      =\qty(\sum_{n=0}^\infty
      \frac{(\vb*\beta\cdot\nabla_{\vb*\beta'})^n}{n!}
      f(\vb*\alpha, \vb*\beta'))
      \bigg|_{\vb*\beta'=\vb*\beta_0}
    \end{equation*}

    Man bemerke dass $\nabla$ einen Operator ist! Die exponential Funktion
    dient hier nur zur vereinfachung der Darstellung!

    Man Taylore um $\vb*r'=0$ (dies heißt, daß $\vb*r\gg\vb*r'$ sodass 
    $\vb*r-\vb*r'\approx\vb*r$, dafür muss $\vb*r$ natürlich weit
    von der Quelle entfernt sein).

    Man definiere nun $f(\vb*r,\vb*r')=f(r_1,r_2,r_3,r_1',r_2',r_3')
    \equiv\rr{1}$
    
    Sodass bis zur 2. Ordnung die Taylorentwicklung für unsere Funktion
    wie folgt aussieht:

    \begin{equation*}
      \rr{1}=f(\vb*r, 0)
      + (\vb*r'\cdot \nabla_{\vb*{\bar{r}'}}) f(\vb*r, \vb*r')\bigg|_{\vb*r'=0}
      + \frac{1}{2}(\vb*r'\cdot\nabla_{\vb*{\bar r'}})^2
      f(\vb*r,\vb*{\bar r}')
      \bigg|_{\vb*{\bar r'}=0}
      + \ldots
    \end{equation*}
    
    \underline{0. Ordnung:}
    \begin{equation*}
      \frac{1}{\abs{\vb*r-\vb*{\bar r'}}}\bigg|_{\vb*{\bar r'}=0} 
      = \frac{1}{\abs{\vb*r}}= \frac{1}{r} 
    \end{equation*}

    \underline{1. Ordnung:}
    \begin{equation*}
      \vb*{\bar r'}\cdot
      \nabla_{\vb*{\bar{r}'}}\frac{1}{\abs{\vb*r-\vb*{\bar{r}'}}}
      \bigg|_{\vb*{\bar r'}=0}
      \stackrel{\text{F1.1}}{=} \frac{\vb*{\bar r'}\cdot\vb*r}{r^3} 
    \end{equation*}
    (Man bemerke dass hier nach $-\vb*{\bar r'}$ ableitet,
    wodurch man das Negative von Frage 1.1 erhält)

    \underline{2. Ordnung:}
    \begin{equation*}
      \begin{split}
      \frac{1}{2}(\vb*{\bar r'}\cdot\nabla_{\vb*{\bar{r}'}})^2
      \frac{1}{\abs{\vb*r-\vb*{\bar r'}}}\bigg|_{\vb*{\bar r'}=0}
      &=
      \frac{1}{2}\qty[r_i'\pdv {\bar r_i'}]\qty[r_j'\pdv{\bar r_j'}]
      \frac{1}{\abs{\vb*r-\vb*{\bar r'}}}
      \bigg|_{\vb*{\bar r'}=0}\\
      &=
      \frac{r'_ir'_j}{2} 
      \frac{\partial^2}{\partial_{\bar r_i'}\partial_{\bar r_j'}} 
      \frac{1}{\abs{\vb*r-\vb*{\bar r'}}}\bigg|_{\vb*{\bar r'}=0}\\
      &\stackrel{\text{F1.1}}{=} 
      \frac{r'_ir'_j}{2} \frac{\partial}{\partial_{\bar r_i'}} 
      \frac{x_j}{\abs{\vb*r-\vb*{\bar r'}}^3}\bigg|_{\vb*{\bar r'}=0}\\
      &=
      \frac{r'_ir'_j}{2} \frac{\partial}{\partial_{\bar r_i'}} 
      \frac{x_j}{\abs{\vb*r-\vb*{\bar r'}}^3}\bigg|_{\vb*{\bar r'}=0}
      \qquad x_j\equiv (r_j-\bar r_j')\\\
      &=
      \frac{r'_ir'_j}{2} 
      \qty(
        \text{-}\frac{3}{2}
        \frac{\text{-}2x_i x_j}{\abs{\vb*r-\vb*{\bar r'}}^5}
        -\frac{\delta_{ij}}{\abs{\vb*r-\vb*{\bar r'}}^3} 
      )\bigg|_{\vb*{\bar r'}=0}\quad\text{(Produktregel)}\\
      &=
      \frac{1}{2}\sum_{i,j=1}^3 \frac{3r_ir_j-\delta_{ij}\vb*r^2}{r^5}
      r_i'r_j'
      \end{split}
    \end{equation*}

    \textit{Eine wichtige Bemerkung:}
    Aus Symmetrie gründe gilt
    \begin{equation*}
      \frac{1}{2}\sum_{i,j=1}^3 \frac{3r_ir_j-\delta_{ij}\vb*r^2}{r^5}
      r_i'r_j'
      =
      \frac{1}{2}\sum_{i,j=1}^3 \frac{3r'_ir'_j-\delta_{ij}\vb*r'^2}{r^5}
      r_ir_j
    \end{equation*}
    
    Wir benutzen im allgemeinen die letzte Definition wenn wir die
    elektrische und magnetische Multipole berechnen!
    

\end{document}
